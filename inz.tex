% !TeX spellcheck = pl_PL
%\documentclass[12pt]{report}
\documentclass[polish]{aghengthesis}

\usepackage{amssymb} % symbol kąta
\usepackage[polish]{babel} % polskie nazwy
\usepackage[T1]{fontenc} % polskie znaki
%\usepackage[margin=1.0in]{geometry} % marginesy
\usepackage[utf8]{inputenc}
\usepackage{listingsutf8} % bloki kodu
\usepackage{lmodern} % font
\usepackage{color} % kolory
\usepackage{indentfirst} % wcięcie w~pierwszej linii paragrafu
\usepackage{graphicx} % obrazy
\usepackage{float} % dla image [H]
\usepackage{amsmath,amsthm,amssymb,mathtools} % matematyka dowód
\usepackage{changepage} % matematyka dowód
\usepackage{siunitx} % wyrównanie do kropki
\usepackage{makecell} % wyrównania nagłówków
\usepackage{hyperref} % bez obwódek wokół linków
\hypersetup{
	colorlinks,
	citecolor=black,
	filecolor=black,
	linkcolor=black,
	urlcolor=magenta
}
\usepackage{caption} % link przenosi do góry obrazka
\usepackage{svg}
\usepackage{enumitem}

% nie rozciągaj,
% TODO usunąć na koniec
%\raggedbottom

% bibliografia
\usepackage[
backend=biber,
]{biblatex}
\addbibresource{inz.bib}
\DeclareFieldFormat{urldate}{dostęp #1}

% enquote (cytaty w~"")
\usepackage[autostyle]{csquotes}
\DeclareQuoteAlias{dutch}{polish}

% mniejsze marginesy wkoło "Spisu treści"
\usepackage{tocloft}
\setlength{\cftbeforetoctitleskip}{20pt}
\setlength{\cftaftertoctitleskip}{30pt}

% bez odstępu w~itemize
\let\tempone\itemize
\let\temptwo\enditemize
\renewenvironment{itemize}{\tempone\setlength{\itemsep}{0cm}}{\temptwo}

\definecolor{lbcolor}{rgb}{0.9,0.9,0.9}  

% Obrazki
% ścieżki do obrazków
\graphicspath{ {img/} }

% obrazek {nazwa.png}{opis}{rozmiar}{H - wymuszanie}
\newcommand{\imgint}[4]{
	\begin{figure}[{#4}]
		\centering
		\includegraphics[width=#3\textwidth]{#1}
		\caption{#2}
		\label{#1}
	\end{figure}
}

% obrazek {nazwa.png}{opis}{rozmiar} (bez wymuszania)
%\newcommand{\imgcs}[3]{\imgint{#1}{#2}{#3}{}}

% obrazek {nazwa.png}{opis} (domyślny rozmiar)
%\newcommand{\img}[2]{\imgcs{#1}{#2}{0.7}}

% obrazek {nazwa.png}{opis}{rozmiar} (wymuszona pozycja)
\newcommand{\imgh}[3]{\imgint{#1}{#2}{#3}{H}}

% dwa obrazki {1}{opis 1}{2}{opis 2}{H - wymuszanie}
\newcommand{\imgintss}[5]{
	\begin{figure}[{#5}]
		\centering
		\begin{minipage}{.45\textwidth}
			\centering
			\includegraphics[width=1\linewidth]{#1}
			\caption{#2}
			\label{#1}
		\end{minipage}%
		\hfill
		\begin{minipage}{.45\textwidth}
			\centering
			\includegraphics[width=1\linewidth]{#3}
			\caption{#4}
			\label{#3}
		\end{minipage}
	\end{figure}
}

% dwa obrazki {1}{opis 1}{2}{opis 2} (bez wymuszania)
%\newcommand{\imgss}[4]{\imgintss{#1}{#2}{#3}{#4}{}}

% dwa obrazki {1}{opis 1}{2}{opis 2} (wymuszona pozycja)
\newcommand{\imghss}[4]{\imgintss{#1}{#2}{#3}{#4}{H}}

% tabela z~wynikami {nazwa.txt}{opis}
\newcommand{\tab}[2]{
	\begin{table}
		\centering
		\caption{#2}
		\vspace{0.3cm}
		\input{#1}
	\end{table}
}

% nagłowek tabeli bold
\renewcommand\theadfont{\bfseries}

% interfejs I2S
\newcommand{\isqs}{$\text{I}^{2}\text{S}$}
% side-set
\newcommand{\sset}{\lstinline|side-set|}

% Listingi kodu
% Style
\lstdefinestyle{default}{
	basicstyle={\ttfamily \footnotesize},
	morekeywords={},
	numbers=left,
	backgroundcolor=\color{lbcolor}
}
\lstdefinestyle{pio}{
	basicstyle={\ttfamily \footnotesize},
	morekeywords={out, side, jmp, set},
	numbers=left,
	backgroundcolor=\color{lbcolor}
}
\lstdefinestyle{scad}{
	basicstyle={\ttfamily \footnotesize},
	morekeywords={module, color, for, translate},
	numbers=left,
	backgroundcolor=\color{lbcolor}
}
\lstdefinestyle{c}{
	basicstyle={\ttfamily \footnotesize},
	morekeywords={while, if, false, DBG_ON, f_read, DBG_OFF},
	numbers=left,
	backgroundcolor=\color{lbcolor}
}

% input listing {style}{caption}{path}
\newcommand{\lstfile}[3]{
	\noindent
	\hspace{0.1\linewidth}
	\begin{minipage}{0.8\linewidth}
		\lstinputlisting[style=#1, caption={#2}, label={#3}]{#3}
	\end{minipage}
	\vspace{0.3cm}
}

% draw.io colors (fg/dark)
\definecolor{io-fg-blue}{HTML}{6c8ebf}
\definecolor{io-fg-green}{HTML}{82b366}
\definecolor{io-fg-orange}{HTML}{d79b00}
\definecolor{io-fg-yellow}{HTML}{d6b656}
\definecolor{io-fg-red}{HTML}{b85450}
\definecolor{io-fg-purple}{HTML}{9673a6}
% draw.io colors (bg/light)
\definecolor{io-bg-blue}{HTML}{dae8fc}
\definecolor{io-bg-green}{HTML}{d5e8d4}
\definecolor{io-bg-orange}{HTML}{ffe6cc}
\definecolor{io-bg-yellow}{HTML}{fff2cc}
\definecolor{io-bg-red}{HTML}{f8cecc}
\definecolor{io-bg-purple}{HTML}{e1d5e7}

\author{Norbert Morawski}
\titlePL{Odbiornik internetowych stacji radiowych}
\titleEN{Receiver of Internet radio stations}

\fieldofstudy{Informatyka}
\supervisor{dr inż. Wojciech Zaborowski}
\date{\the\year}

\setcounter{tocdepth}{1}

\begin{document}
\lstset{inputencoding=utf8, basicstyle=\ttfamily}

\maketitle

\addtocontents{toc}{\protect\thispagestyle{empty}}
\tableofcontents

\cleardoublepage
\chapter{Cel prac i~wizja produktu}
	Tradycyjne radio powstało na początku XX wieku i~do dzisiaj audycje radiowe cieszą się wysoką popularnością. Według podsumowania serwisu iloveradio.pl\textsuperscript{\cite{iloveradio_stats}} w~2022~roku 65\% Polaków słuchała radia codziennie, a~prawie 87\% co najmniej raz w~tygodniu uruchamiało swoje radioodbiorniki. 
	$ $\\
	
	Uwarunkowania prawno-ekonomiczne, związane z~tradycyjną emisją programu radiowego i~postęp w~dziedzinie technologii komunikacyjnych sprawiły, że rozwinęło się radio Internetowe, którego popularność i~słuchalność regularnie wzrasta. Według badań Krajowego Instytutu Mediów łączny zasięg dzienny stacji słuchanych przez Internet w~2022~wyniósł 6,3\%\textsuperscript{\cite{kim_2022}}. W~analogicznym okresie 2023~roku wynik podwyższył się do 7,6\%\textsuperscript{\cite{kim_2023}}.
	Związane jest to z~kilkoma czynnikami. Radio internetowe często oferuje dużo lepszą technicznie jakość odbieranego dźwięku. Do wyboru jest znacznie większa liczba rozgłośni, wśród nich są rozgłośnie tematyczne, emitujące określony typ audycji czy rodzaj muzyki. Dodatkowo, radio internetowe nie ma ograniczeń geograficznych związanych z~odbiorem, jak ma to miejsce w~przypadku tradycyjnego radia. Przez Internet można słuchać stacji nadającej z~dowolnego miejsca na świecie.
	
	\section{Opis problemu}
			Niestety, typowy odbiornik radia naziemnego nie umożliwia odbierania audycji nadawanych przez rozgłośnie internetowe.
			Do ich odbioru wykorzystywane jest najczęściej odpowiednie oprogramowanie pracujące na komputerze lub smartfonie.
			Może się to wiązać z~pewnymi niedogodnościami: nie zawsze istnieje możliwość uruchomienia takiej aplikacji na dostępnym urządzeniu, obsługa takiej aplikacji może być w~pewnych okolicznościach niewygodna lub utrudniona, wreszcie samo urządzenie może nie być odpowiednio wyposażone np. w~głośniki umożliwiające odsłuch odbieranej stacji lub użytkownik nie chce narażać urządzenia na niekorzystne warunki otoczenia. Ponad to, istnieje grupa użytkowników o~stosunkowo niskiej kulturze technicznej, dla których korzystanie z~komputera lub smartfona jest zadaniem trudnym, natomiast bez problemu radzą sobie z~obsługą stosunkowo prostych urządzeń, do których można zaliczyć tradycyjny radioodbiornik. Powstał zatem pomysł zaprojektowania i~zbudowania, opisanego w~niniejszej pracy, przenośnego urządzenia, wbudowanego realizującego funkcję odbiornika Internetowych stacji radiowych.
		
	\section{Rola produktu}
		Budowany w~ramach tej pracy odbiornik radia internetowego jest przeznaczony do osób z~małą wiedzą techniczną. Jego obsługa powinna być prosta i~intuicyjna. Do korzystania z~niego nie powinny być wymagane żadne dodatkowe elementy (np.~zewnętrzne głośniki czy specjalna ładowarka).
		Produkt ma być kompletnym urządzeniem gotowym do odbioru internetowych stacji radiowych. Całość będzie zamknięta w~obudowie. Na rysunku~\ref{1/radio_proj_trimmed} przedstawiony został projekt wizualny radia.
%		Rysunek~\ref{1/PicoRadio_blocks} przedstawia uproszczony schemat blokowy urządzenia.
		
		\imgh{1/radio_proj_trimmed}{Projekt wizualny radia (z zachowanymi proporcjami)}{0.7}
%		\imgcs{1/PicoRadio_blocks}{Schemat blokowy radia}{1}
		
	\section{Współpracujące systemy}
		Oczywistym współpracującym systemem jest w~przypadku tego projektu serwer strumieniujący radio internetowe.
%		Standardem kodowania wykorzystywanym przez serwery jest format MP3 lub AAC/AAC+.
		Drugim współpracującym systemem jest baza danych zbierająca informacje o~dostępnych stacjach radia internetowego na świecie. Baza taka powinna umożliwiać zapytania poprzez udokumentowane i~dostępne API. List baz brana pod uwagę przy projektowaniu urządzenia:
		\begin{itemize}
			\item Radio-browser\textsuperscript{\cite{radio-browser}}
			\item fmstream\textsuperscript{\cite{fmstream}}
		\end{itemize}
		
		Druga baza, choć obszerniejsza (86,000 a~38,578 stacji), nie udostępnia interfejsu dla wszystkich. Autor życzy sobie osobistej prośby o~dostęp, która na ten moment została wysłana i~czeka na odpowiedź.
		
	\section{Wymagania}
		Po konsultacji z~promotorem, ustalony został zakres wymagań funkcjonalnych. Przedstawione zostały one poniżej. Główny nacisk położony został na funkcje urządzenia.
		Przygotowanie odpowiedniej dokumentacji technicznej było równie ważnym aspektem realizacji projektu.
	
%		Główną funkcją budowanego urządzenia ma być możliwość odtwarzania radia internetowego.
%		Dodatkowo użytkownik będzie mógł wyszukiwać nowe stacje i~zapisywać je na liście ulubionych. Przewidziano także opcje odtwarzania lokalnych plików MP3. Poniżej zostały one 
%		Poniżej zostały przedstawione wymagane funkcjonalności, jak również te dodatkowe.
		
		\subsection{Wymagane funkcjonalności}
			\begin{itemize}
				\item odtwarzanie stacji internetowych,
				\item wstrzymywanie odtwarzania,
				\item prezentacja na zintegrowanym wyświetlaczu aktualnego stanu urządzenia,
				\item regulacja głośności odtwarzania,
				\item odbiór i~reprodukcja dźwięku stereo,
				\item wyszukiwanie nowych stacji,
				\item lista ulubionych stacji.
			\end{itemize}
			
			\paragraph{Dodatkowe funkcjonalności (niewymagane)}
			\begin{itemize}
				\item odtwarzanie plików z~dołączanego do urządzenia nośnika wymiennego (karta SD).
			\end{itemize}
		
			\paragraph{Oczekiwany produkt}
			\begin{itemize}
				\item W~pełni działający prototyp urządzenia realizującego funkcje stereofonicznego odbiornika internetowych stacji radiowych,
				\item Dokumentacja techniczna pozwalająca na zbudowanie urządzenia oraz rozwój jego oprogramowania układowego.
			\end{itemize}
		
		\subsection{Wymagania niefunkcjonalne}
			Urządzenie ma być wykonane z~wykorzystaniem mikrokontrolera, lub płytki rozwojowej opartej o~mikrokontroler.
			Do pobierania strumienia danych audio z~serwera wykorzystać należy  bezprzewodowe połączenie z~Internetem oparte o~technologię Wi-Fi.
			Produkt musi także poradzić sobie z~obsługą odpowiedniego formatu kodowania dźwięku.
			Rysunek~\ref{3/rb_chart_new} przestawia procentowe wykorzystanie kodeków przez stacje radiowe (wg wyszukiwarki \textit{Radio Browser}\textsuperscript{\cite{radio_browser_codecs}}).
			
			\imgh{3/rb_chart_new}{Udział kodeków w~stacjach radiowych}{0.8}
			
			Najwięcej, około 66\% wszystkich i~60\% polskich stacji, używa kodeka MP3. Wybór padł na właśnie ten format.
			Zapewni on kompatybilność urządzenia z~największą liczbą dostępnych stacji radia internetowego.
			\pagebreak
%			$ $\\
			
			Zdecydowano się też na miniaturyzację wyświetlacza i~zastosowanie do obsługi urządzenia kilku przycisków. Z~uwagi na małą moc obliczeniową mikrokontrolera, regulacja głośności odbywać się będzie analogowym potencjometrem.
		
	\section{Przegląd konkurencyjnych rozwiązań}
		Rosnąca popularność internetowych stacji radiowych sprawiła, że na rynku dostępne są gotowe urządzenia odbiorcze.
		Na rysunku~\ref{1/radio_adapter} przedstawiony został komercyjny adapter radia internetowego (odbiornik przeznaczony do podłączenia zewnętrznego wzmacniacza i~głośników). Ceny takich urządzeń zaczynają się od ok. 330~zł za produkt powystawowy. 
		\imgh{1/radio_adapter}{Adapter radia internetowego DUAL~IR~3A\textsuperscript{\cite{radio_adapter}}}{0.7}
		
		Innym przykładem, bardziej kompaktowego odbiornika jest pokazany na rysunku~\ref{1/radio_tanie} odbiornik STERNRADIO IR~2~firmy TechniSat.
		\imgh{1/radio_tanie}{Odbiornik STERNRADIO IR~2~firmy TechniSat\textsuperscript{\cite{radio_tanie}}}{0.7}
		Jest to urządzenie przenośne, monofoniczne, zasilane z~wbudowanego akumulatora. Jego obecny koszt zakupu wynosi około 240~zł.
		
		\pagebreak
		Kolejnym przykładem konkurencyjnego rozwiązania jest urządzenie przestawione na rysunku~\ref{1/radio_drogie}.
		\imgh{1/radio_drogie}{Stereofoniczny odbiornik radia internetowego Blaupunkt IR10BT\textsuperscript{\cite{radio_drogie}}}{0.7}
		Produkt obsługuje dźwięk stereofoniczny. Obecny koszt jego zakupu to 380~zł. W~porównaniu z~urządzeniem firmy TechniSat oferuje dodatkowo obsługę łączności Bluetooth oraz integrację z serwisem Spotify.
		$ $\\
		
		Obok rozwiązań komercyjnych w~Internecie znaleźć można rozmaite projekty typu \enquote{zrób to sam} opisujące konstrukcje odbiorników radia internetowego. Przykładem takiego rozwiązania jest konstrukcja\textsuperscript{\cite{rpi_sbc_radio}} oparta o~miniaturowy komputer jednopłytkowy Raspberry Pi model B, pracujący pod kontrolą systemu operacyjnego Linux. System operacyjny został skonfigurowany w~taki sposób aby automatycznie uruchamiana była, napisana w~języku Python, aplikacja, która łączy się z~jednym z~zakodowanych na stałe serwerów stacji radiowej i~uruchamia odtwarzanie wysyłanego przez nią strumienia. Całość jest zamknięta w~obudowie pozyskanej z~klasycznego przenośnego radioodbiornika.
		$ $\\
		
		Odbiornik radia internetowego można zbudować wykorzystując gotowe, oparte o~mikrokontrolery rozbudowane zestawy uruchomieniowe. W~ramach zrealizowanej w~Instytucie Informatyki AGH pracy inżynierskiej\textsuperscript{\cite{apd_radio2}} powstało urządzenie oparte o~pokazaną na rysunku~\ref{1/botland_stm} płytkę rozwojową STM32F746G-Disco\textsuperscript{\cite{stm32_disco}}.
		
		\imgh{1/botland_stm}{Płytka rozwojowa STM32F746G-Disco\textsuperscript{\cite{botland_stm}}}{0.6}
		
		Wspomniany zestaw uruchomieniowy zawiera m.in. wbudowany, duży, kolorowy, wyposażony w~moduł ekranu dotykowego wyświetlacz LCD, kodek audio oraz interfejs Ethernet. Znajdujący się w~nim mikrokontroler jest na tyle wydajny, że bez problemu programowo dekoduje strumienie MP3 a~duży wyświetlacz pozwolił Autorom na stworzenie atrakcyjnego interfejsu użytkownika. Odbiornik ten ma jednak cechy urządzenia stacjonarnego. Wymaga dołączenia zewnętrznego zasilania, głośników oraz przewodowego połączenia do sieci lokalnej. Niestety cena samego zestawu oscylująca aktualnie w~okolicach 400~zł jest mało atrakcyjna.
		$ $\\
		
		Jednym z~wymagań niefunkcjonalnych realizowanego w~ramach tej pracy odbiornika jest wykorzystanie bezprzewodowego połączenia Wi-Fi do odbierania strumienia danych z~serwera stacji radiowej. Podobne rozwiązanie zrealizowane zostało w~ramach projektu Mini Web Radio\textsuperscript{\cite{esp32_radio}}. Odbiornik ten zbudowano wykorzystując zestaw uruchomieniowy Lolin32 Lite oparty o~układ SoC ESP32. Interesującą cechą rodziny tych układów jest zintegrowany moduł komunikacji Wi-Fi. Między innymi dzięki temu powstała niewielka, przenośna, zasilana bateryjnie konstrukcja. Pewną niedogodnością w~jej użytkowaniu może być jednak fakt, interfejs użytkownika został maksymalnie uproszczony i~pozwala jedynie na regulacje głośności odtwarzanej stacji. Natomiast zaawansowana konfiguracja, obejmująca m.in. wprowadzenie adresu serwera stacji radiowej odbywa się za pośrednictwem przeglądarki internetowej.
		
		%		Jest to relatywnie rozbudowany zestaw uruchomieniowy. Zawiera wbudowany wyświetlacz, złącze Ethernet, jak również przetwornik cyfrowo-analogowy. Jednak brakuje w~nim obsługi technologii Wi-Fi czy zasilania akumulatorowego, co uniemożliwia zastosowanie go jako przenośnego odtwarzacza radia internetowego.
		%		$ $\\
		%		W~celu obsługi bezprzewodowego połączenia z~Internetem można użyć mikrokontrolera z~wbudowaną jego obsługą np. ESP32. Projekt Mini~Web~Radio\textsuperscript{\cite{esp32_radio}} używa tego mikrokontrolera umieszczonego na płytce z~obsługą ładowania baterii. Jest to kompaktowe rozwiązane które dodatkowo umożliwia miniaturyzację urządzenia.
	
	\section{Analiza technologiczna}
		Inspirując się projektem Mini~Web~Radio oraz sugestiami promotora, do realizacji projektu wybrany został moduł uruchomieniowy Raspberry~Pi~Pico~W.
		Wykorzystany do jego budowy mikrokontroler RP2040\textsuperscript{\cite{rp2040}} zawiera dwa rdzenie \textit{Cortex M0+}, co umożliwi zrównoleglenie niektórych zadań.
		Układ zawiera także wiele peryferiów. Pomogą one m.in. w~implementacji protokołu przesyłu dźwięku \isqs{} oraz odciążeniu procesora poprzez wykorzystanie transferów DMA. Na płytce modułu znajduje się również układ radiowy umożliwiający komunikację za pośrednictwem standardu Wi-Fi.
		
		\subsection{Dekodowanie formatu MP3}
			Jednym z~trudniejszych zagadnień związanych z~budową odbiornika internetowych stacji radiowych jest dekodowanie strumienia danych w~formacie MP3. Ze względu na złożoność metody kodowania, ograniczenia patentowe i~krótki czas realizacji projektu odrzucony został pomysł samodzielnego napisania oprogramowania dekodującego na rzecz wykorzystania dostępnych bibliotek.
			
			Jedną z~popularnych bibliotek oferujących dekodowanie formatu MP3 jest \lstinline|minimp3|\textsuperscript{\cite{minimp3}}. Sprawdza się ona doskonale w~aplikacjach dedykowanych dla komputerów osobistych i~wydajnych mikrokontrolerów, jednak jej wadą jest to, że wykorzystuje arytmetykę zmiennoprzecinkową. Ponieważ mikrokontroler RP2040 nie wspiera sprzętowo operacji zmiennoprzecinkowych, istniała uzasadniona obawa, że nie poradzi on sobie z~dekodowaniem strumienia MP3 w~czasie rzeczywistym. W~takiej sytuacji konieczne okazało się użycie  biblioteki \lstinline|helixmp3|\textsuperscript{\cite{helixmp3_repo}} która implementuje dekoder MP3 przy użyciu arytmetyki stałoprzecinkowej.
			$ $\\
			
			Szczegóły związane z~implementacja procesu dekodowania strumienia MP3 omówione zostały w~dalszej części pracy.
		
		\subsection{Oprogramowanie}
			Oprogramowanie zostanie napisane w~języku C++ z~wykorzystaniem środowiska \lstinline|pico-sdk|\textsuperscript{\cite{pico_sdk}}.  W~środowisko brak wbudowanego centralnego systemu zarządzającego przepływem informacji oraz zadaniami, zatem do projektu dołączone zostało jądro systemu czasu rzeczywistego FreeRTOS\textsuperscript{\cite{freertos_kernel}}
	
	\section{Ryzyko}
		Podczas budowy prototypu odbiornika i~tworzenia jego oprogramowania układowego może wystąpić szereg różnych problemów, które mogą opóźnić prace projektowe.
		$ $\\
		
		Pierwszym czynnikiem opóźniającym mogą stać się problemy we współpracy między sobą poszczególnych komponentów sprzętowych odbiornika, prowadzące do niedziałania lub błędnego działania urządzenia. Ich usuniecie może wymagać czasu niezbędnego do ustalenia przyczyn i~znalezienia rozwiązania. Minimalizacja oddziaływania tego czynnika polegać będzie na odpowiednim dobraniu komponentów odbiornika i~jak najszybszym przeprowadzeniu testów wstępnie weryfikujących poprawność ich współpracy. W~przypadku wystąpienia problemów konieczne będzie znalezienie innych, poprawnie współpracujących komponentów i~wprowadzenie niezbędnych zmian konstrukcyjnych.
		$ $\\
		
		Drugi z~istotnych czynników ryzyka to problemy wydajnościowe związane z~jednoczesnym odbieraniem strumienia sygnału stacji radiowej, jego dekodowaniem i~odtwarzaniem. Mikrokontroler modułu Raspberry Pi Pico W~taktowany jest zegarem 133~MHz, co  w~porównaniu chociażby z~komputerami jednopłytkowymi Raspberry Pi jest wartością niewielką. Może okazać się że dostępne biblioteki dekodujące nie pozwolą na uzyskanie odpowiedniej wydajności procesu dekodowania co skutkować będzie niepoprawnym odtwarzaniem sygnału stacji. Konieczne wtedy może się okazać znalezienie innej biblioteki dekodującej lub w~skrajnej sytuacji konieczna będzie zmiana założeń projektowych. Minimalizacja wpływu tego czynnika polegać będzie na takim zorganizowaniu prac, aby jak najwcześniej przeprowadzić testy pozwalające stwierdzić czy wydajność procesu dekodowania jest wystarczającą. 
		$ $\\
		
		Kolejnym istotnym czynnikiem ryzyka jest rozmiar pamięci RAM mikrokontrolera RP2040.  Może się okazać, że 264kB RAM to zbyt mało by umieścić w~niej wszystkie struktury danych niezbędne do  prawidłowego działania oprogramowania układowego. Czynnikiem zaradczym będzie w~tym przypadku oszczędne gospodarowanie pamięcią i~monitorowanie stopnia jej wykorzystania przez cały czas prac nad kodem oprogramowania układowego.
		$ $\\
		
		Niezakłócone odtwarzanie audycji radiowej związane jest ze stabilnym przepływem danych z~serwera rozgłośni do odbiornika. Może się okazać, że domyślna konfiguracja komunikacji sieciowej spowoduje trudne do zidentyfikowania zakłócenia stabilnego przepływu danych. Ustalenie przyczyny oraz znalezienie rozwiązania wymagać będzie przeprowadzenia wielu testów i~może okazać się poważnym wyzwaniem.
		$ $\\
		
		Istnieje niebezpieczeństwo, że projektując architekturę oprogramowania układowego na skutek zaniechania lub niewłaściwych założeń, popełnione zostaną błędy, których usunięcie wymagać będzie wprowadzenia do niej zasadniczych zmian a~ich pozostawienie ograniczy w~sposób istotny funkcjonalności odbiornika. 
		$ $\\
		
		Na każdym etapie prac istnieje ryzyko przypadkowego uszkodzenia lub zniszczenia prototypu, co spowoduje całkowite wstrzymanie prac do czasu jego odbudowy lub naprawy. Jedyną możliwością minimalizacji wpływu takiego zdarzenia na harmonogram prac jest posiadanie drugiego egzemplarza prototypu lub co najmniej posiadanie dodatkowego egzemplarza kluczowego elementu jakim jest moduł Rasperry Pi Pico W.
	
	\section{Słownik pojęć}
	\begin{itemize}
		\setlength{\itemsep}{0.1cm}
		\item MP3 (\textit{MPEG-1 Audio Layer 3}) -- popularny format zapisu dźwięku; implementuje stratną kompresję,
		\item AAC/AAC+ (\textit{Advanced Audio Coding}) -- inny format dźwięku z~kompresją stratną,
		\item Li-ion -- akumulatory Litowo-jonowe (ładowalne),
		\item Raspberry Pi Pico W~-- płytki rozwojowe oparte o~mikrokontroler RP2040 (produkowane przez Raspberry Pi Foundation), wersja z~wbudowanym modułem komunikacyjnym Wi-Fi
		\item \textit{Cortex M0+} -- rdzeń procesora, zaprojektowany przez firmę ARM i~sprzedawany innym podmiotom jako własność intelektualna; relatywnie nieskomplikowany i~zajmujący małą powierzchnię krzemu, jednak przez to najmniej wydajny,
%		\item PWM (ang. \textit{Pulse Width Modulation}) -- 
		\item DMA (ang. \textit{Direct Memory Access}) -- podsystem mikrokontrolera, który pozwala na automatyczne transfery danych do pamięci, bez ingerencji jednostki centralnej,
		\item PIO (ang. \textit{Programmable Input Output}) -- programowalne maszyny stanów zawarte w~układzie RP2040,
		\item DAC (ang. \textit{Digital to Analogue Converter}) -- przetwornik cyfrowo-analogowy,
		\item \isqs{} (ang. \textit{Inter-IC Sound}) -- protokół przesyłu cyfrowego dźwięku opracowany w~1986~roku przez firmę Philips,
		\item PCM (ang. \textit{Pulse-Code Modulation}) -- metoda reprezentacji sygnału analogowego, poprzez próbkowanie i~zapis chwilowych wartości amplitudy dźwięku w~postaci cyfrowej,
		\item SIO (ang. \textit{Single-cycle IO}) -- blok mikrokontrolera RP2040 zapewniający synchronizację obu jego rdzeni, komunikację miedzy nimi oraz atomiczny dostęp do pinów wejścia/wyjścia,
		\item TCP (ang. \textit{Transmission Control Protocol}) -- protokół zapewniający integralność danych przesyłanych w~Internecie, oraz zapobiegający przeciążeniu odbiornika nadmierną ilością danych,
		\item TLS (ang. \textit{Transport Layer Security}) -- protokół zapewniający szyfrowanie i~poufność danych przesyłanych w~Internecie,
	\end{itemize}

\cleardoublepage
\chapter{Zakres funkcjonalności}
	W~rozdziale drugim skupiono się na krótkim wprowadzeniu do sprzętowej architektury odbiornika. Przedstawiony został także opis funkcjonalności produktu oraz projekt interfejsu użytkownika.\\
	
	\section{Sprzętowa architektura systemu}
			Architekturę systemu realizującego funkcje odbiornika internetowych stacji radiowych przedstawiono na rysunku~\ref{2/PicoRadio-hw-blocks}. Składa się on z~następujących elementów:
			\begin{itemize}
				\item Płytka rozwojowa z~mikrokontrolerem (kolor \textcolor{io-fg-orange}{pomarańczowy}),
				\item sekcja audio (kolor \textcolor{io-fg-red}{czerwony}),
				\item podsystem interfejsu użytkownika (kolor \textcolor{io-fg-purple}{purpurowy}),
				\item złącza zewnętrzne i~sekcja zasilania (kolor \textcolor{io-fg-blue}{niebieski}),
				\item współpracujące źródła danych (kolor \textcolor{io-fg-green}{zielony}).
			\end{itemize}
			
			\imgh{2/PicoRadio-hw-blocks}{Sprzętowa architektura systemu}{0.95}
			
			\pagebreak
			Bazę odtwarzacza stanowi płytka rozwojowa Raspberry Pi Pico~W oparta o~mikrokontroler RP2040. Zawiera ona ponadto 2MB nieulotnej pamięci typu \textit{FLASH} (przeznaczonej na oprogramowanie układowe i~dane użytkownika) oraz układ komunikacji Wi-Fi. Do niej dołączony jest przetwornik audio DAC który wraz dodatkowymi komponentami (potencjometry regulacji głośności, wzmacniacze mocy i~głośniki) tworzą tor audio. Role interfejsu użytkownika pełni ekran LCD o~przekątnej 1,8~cala oraz 5~przycisków chwilowych typu pushbutton. Do płytki dołączone jest również gniazdo dla kart SD. Całość jest zasilana z~ogniwa litowo-jonowego współpracującego z~układem jego monitorowania i~ładowania. 
			$ $\\
			
			Zasada działania urządzenia jest następująca.
			Zaraz po uruchomieniu, urządzenie próbuje połączyć się z~jedną ze skonfigurowanych sieci Wi-Fi.
			Jeśli zakończy się to sukcesem użytkownik ma możliwość wyboru stacji radiowej do odbioru. W~trakcie odtwarzania urządzenie odbiera skompresowany strumień danych audio, który jest następnie dekodowany przez oprogramowanie układowe do formatu PCM. Zdekodowane dane są przesyłane poprzez interfejs \isqs{} do stereofonicznego przetwornika audio, który zamienia je w~sygnał analogowy. Z~wyjść przetwornika sygnał ten trafia na, zbudowany z~podwójnego potencjometru logarytmicznego, układ regulacji głośności i~dalej do układów wzmacniacza mocy sterującego głośnikami kanału lewego i~prawego.
			Użytkownik może w~dowolnym momencie zatrzymać odtwarzanie lub zmienić odtwarzaną stację na inną wybierając ją z~pośród ulubionych lub wyszukując na liście dostępnych stacji.
			$ $\\
			
			Jeśli nie uda się nawiązać połączenia z~siecią Wi-Fi, próby odtwarzania stacji radiowych są blokowane poprzez wyświetlenie stosownego komunikatu.
			Jednak wtedy użytkownik ma możliwość uruchomienia odtwarzania plików z~dołączonej karty SD. Po wybraniu odpowiedniego pliku, sposób działania jest identyczny jak w~trybie odtwarzania stacji radiowej, z~tą różnicą że strumień danych pobierany jest pliku znajdującego się na karcie. 
			W~każdym momencie użytkownik ma możliwość skonfigurowania połączenia internetowego, poprzez dodanie nowej sieci Wi-Fi.
			Musi jednak najpierw przerwać odtwarzanie i~otworzyć ekran ustawień.
	
	\section{Projekt interfejsu}
		\label{sec:ui}
		Najważniejszym elementem planowania odbiornika z~perspektywy użyteczności dla klienta jest projekt wygodnego i~intuicyjnego interfejsu użytkownika. Komplikacją jest fakt, że urządzenie będzie miało do dyspozycji tylko niewielki wyświetlacz i~kilka przycisków.
		Należy więc zadbać, żeby przy tych ograniczonych możliwościach udało się przekazać niezbędne informacje o~stanie urządzenia oraz zapewnić użytkownikowi prostą i~ergonomiczną jego obsługę.
		$ $\\

		Do zaprojektowania ekranów, z~których następnie powstanie interfejs użytkownika, posłużyło narzędzie Figma\textsuperscript{\cite{figma}}. Na rysunku~\ref{2/interface/PicoRadio-flow} przedstawiono jak wyglądają poszczególne okna oraz jak użytkownik może przemieszczać się  pomiędzy nimi.
		
		\imgh{2/interface/PicoRadio-flow}{Przemieszczanie się pomiędzy ekranami}{0.9}
		
		Najczęściej wyświetlanym ekranem jest ekran odtwarzacza radia internetowego. Użytkownik widzi wówczas umieszczoną na ekranie nazwę aktualnie odtwarzanej stacji a~pod nią tytuł audycji lub wykonawcę i~tytuł aktualne nadawanego utworu.
		$ $\\
		
		W~górnej części wyświetlacza widoczna jest nazwa aktualnie otwartego ekranu. Informuje ona o~trybie pracy urządzenia.
		W~zależności od aktualnie wyświetlanego ekranu, obok jego nazwy, znajdują się ikony statusu, a~w dolnej części ikony akcji.
		
		\paragraph{Ikony statusu}
			Przy projektowaniu interfejsów poczyniono założenie, że ikony statusowe (połączenie WiFi/włożona karta SD/etc.) znajdować się będą na górze ekranu. Nie będą one klikalne.
			
		\paragraph{Ikony akcji}
			Na ekranach w~dolnej części umieszczono ikony akcji. Jest to np. ikona wstecz lub wyszukaj. Użytkownik dzięki nim będzie mógł przeprowadzać dodatkowe akcje.
		
		$ $\\
		Przy wyświetlaniu listy (np. wyników wyszukiwania lub listy ulubionych stacji) kliknięcie w~daną pozycję powoduje zaakceptowanie jej i~otworzenie następnego ekranu.
		$ $\\
		
		Na rysunkach \ref{2/interface/real1}, \ref{2/interface/real2} umieszczono zdjęcia przedstawiające wygląd interfejsu na rzeczywistym wyświetlaczu. W~sekcji~\ref{sec:uinav} omówiono sposób nawigacji z~wykorzystaniem przycisków.
		\imghss{2/interface/real1}{Wyszukiwanie stacji}{2/interface/real2}{Wyniki wyszukiwania}
	
	\section{Interakcja użytkownika z~interfejsem produktu}
		\label{sec:uinav}
		Poniżej przedstawione zostały wybrane scenariusze poruszania się użytkownika w~obrębie zaprojektowanego interfejsu.
		$ $\\
		
		Na rysunku~\ref{2/btn/1} widać jak użytkownik aby poruszyć się w~dół listy stacji naciska dolny przycisk. Powrót do pozycji \textit{Radio 1} jest oczywiście możliwy poprzez naciśnięcie górnego przycisku.
		\imgh{2/btn/1}{Krok w~dół na liście}{0.7}
		
		Po dojściu do czwartej pozycji na liście, jeżeli na liście jest więcej niż 4~stacje, po naciśnięciu dolnego przycisku, zgodne z~rysunkiem~\ref{2/btn/2}, kursor zostanie przeniesiony do piątej stacji. Należy zwrócić uwagę na zmianę położenia paska przewijania (po prawej stronie).
		\imgh{2/btn/2}{Przewinięcie listy stacji do góry}{0.7}
		
		Jeżeli wskaźnik znajduje się na końcu listy (pasek przewijania w~skrajnie dolnej pozycji), po naciśnięciu przycisku w~dół, kursor znajdzie się w~pasku ikon akcji. Sytuację tę przedstawia rysunek \ref{2/btn/3}. Powrót na listę stacji jest możliwy przy pomocy górnego przycisku.
		\imgh{2/btn/3}{Przejście z~listy do ikon akcji}{0.7}
		
		Poruszanie się w~obrębie jednego wiersza zostało zrealizowane poprzez przyciski \mbox{lewo/prawo}. Przedstawiają to rysunki \ref{2/btn/4r} i~\ref{2/btn/4l} na przykładzie dwóch ikon akcji.
		\imghss{2/btn/4r}{Przejście w~prawo}{2/btn/4l}{Przejście w~lewo}
		
		Przycisk środkowy służy do wykonania akcji. Jeżeli kursor jest ustawiony na liście, naciskając go, możemy rozpocząć odtwarzanie wybranej stacji (rysunek \ref{2/btn/5}). Nastąpi wówczas przejście do ekranu \textit{Radio}.
		\imgh{2/btn/5}{Akceptacja wyboru}{0.7}
				
		Czasami w~komunikacji z~serwerem może wystąpić błąd. Takie sytuacje są zgłaszane użytkownikowi poprzez ekrany błędów widoczne na rysunkach \ref{2/err1} i~\ref{2/err2} (czerwone tło). Aby wyjść z~takiego ekranu wystarczy nacisnąć dowolny przycisk.
		\imghss{2/err1}{Błąd odtwarzania}{2/err2}{Błąd ładowania stacji}

\cleardoublepage
\chapter{Wybrane aspekty realizacji}
	W~tym rozdziale opisano ukończony produkt.
	Przedstawiono jak funkcjonuje część sprzętowa oraz programowa projektu. Opisano także poszczególne bloki funkcjonalne.
	
	\section{Część sprzętowa}
		\label{sec:hw}
		Opisane w~niniejszej pracy urządzenie jest kompletnym, przenośnym, wyposażonym we własne zasilanie akumulatorowe  odbiornikiem internetowych stacji radiowych. Do pracy nie wymaga żadnych dodatkowych elementów w~postaci głośników lub zewnętrznego wzmacniacza.
		Projekt schematu i~płytki PCB powstały w~programie KiCAD\textsuperscript{\cite{hw_kicad}}.
	
		\subsection{Schemat}
%			Schemat całości systemu został pokazany na rysunku~\ref{3/hw_kicad_sch}. Podzielony jest on na 6~funkcjonalnych bloków, które zostały opisane poniżej.
%			\imgh{3/hw_kicad_sch}{Schemat elektryczny radia}{1}

			Schemat całości systemu został podzielony na 6~funkcjonalnych bloków, które zostały opisane poniżej.
			
			\subsubsection{Sekcja zasilania}
				Rysunek~\ref{3/hw_kicad_sch_power} przedstawia zaprojektowaną sekcję zasilania.
				\imgh{3/hw_kicad_sch_power}{Schemat elektryczny radia: Sekcja zasilania}{0.95}
				
				Urządzenie posiada własne zasilanie oparte o~dwa ogniwa litowo-jonowe typu 18650~o~sumarycznej pojemności 4Ah i~znamionowym napięciu 3.6V. Na schemacie ogniwa zaznaczono wspólnym symbolem \lstinline|BT1|. W~roli układu ładowania i~zabezpieczeń został wykorzystany gotowy moduł z~układem TP4056\textsuperscript{\cite{hw_chg}} (schemat ozn.~\lstinline|B3|). Dzięki niemu, produkt nie wymaga stosowania specjalistycznej ładowarki, a~same ogniwa są zabezpieczone przed nadmiernym rozładowaniem i~przeładowaniem. Zgodnie z~najnowszymi trendami, zasilanie z~zewnątrz dostarczane jest przez złącze USB-C. Użytkownik może wykorzystać do tego dowolną kompatybilną ładowarkę o~prądzie minimalnym 1A.
				$ $\\
				
				Dodatkowe elementy: tranzystor PMOS \lstinline|Q1|, Dioda Shottky'ego \lstinline|D1| i~rezystor \lstinline|R5| tworzą układ zabezpieczający ogniwa \lstinline|BT1| przed przeładowaniem podczas  zasilania urządzenia z~zewnętrznego źródła. Konieczność ich użycia została szczegółowo wyjaśniona w~\cite{hw_load_sw}. W~sekcji zasilania umieszczono ponad to dzielniki pomiarowe wykorzystane do monitorowania jej pracy oraz przetwornicę wytwarzającą napięcie +5V (schemat ozn.~\lstinline|B4|), która w~przyszłości posłużyć może do zasilania zewnętrznego modemu sieci komórkowej.
				$ $\\
				
				Napięcie 3.3V, niezbędne do zasilania wyświetlacza oraz karty SD pobierane jest z~wyjścia stabilizatora 3.3V znajdującego się na płytce Raspberry Pi Pico~W.
				
			\subsubsection{Panel LCD, przyciski, karta SD}
				Na rysunku~\ref{3/hw_kicad_sch_perip} przedstawiono schemat dołączenia modułów realizujących funkcje wejścia/wyjścia w~odbiorniku.
				\imgh{3/hw_kicad_sch_perip}{Schemat elektryczny radia: peryferia wejścia/wyjścia}{0.95}
				
				Wyświetlacz LCD oraz gniazdo karty SD dołączone są bezpośrednio do wyprowadzeń modułu Raspberry Pi Pico~W. Niezbędne do ich zasilania napięcie 3.3V jest również pobierane z~Pi Pico~W.
				$ $\\
				
				Moduł klawiatury składa się z~pięciu przycisków membranowych typu pushbutton zwierających do masy odpowiednie wyjście. Zrealizowany jest jako osobna płytka (zostanie zamontowany na przednim panelu obudowy).  Do wszystkich sygnałów pochodzących z~wejść będących stykami mechanicznymi dołączone zostały równolegle kondensatory 100nF służące do eliminacji zjawiska drgania styków.
				\pagebreak
				
			\subsubsection{Tor audio}
				Rysunek~\ref{3/hw_kicad_sch_audio} przedstawia schemat toru analogowego, który jest odpowiedzialny za odtwarzanie dźwięku.
				\imgh{3/hw_kicad_sch_audio}{Schemat elektryczny radia: Tor audio}{0.95}
				
				W~torze audio zastosowano gotowy moduł oparty o~scalony, stereofoniczny przetwornik PCM5102A\textsuperscript{\cite{hw_dac}}. Dane do przetwornika dostarczane są z~modułu Raspberry Pi Pico~W przez interfejs \isqs{}. Moduł przetwornika posiada własne stabilizatory napięcia, w~związku z~tym zasilany jest napięciem 3.7V pobieranym bezpośrednio z~ogniw.
				$ $\\
				
				Z~wyjścia przetwornika sygnały poszczególnych kanałów są wprowadzane na dwustopniowy dzielnik napięcia zrealizowany na dwóch podwójnych, sprzężonych potencjometrach obrotowych. Pierwszy stopień regulacji dostępny jest wewnątrz urządzenia i~służy do ograniczenia, na etapie uruchamiania odbiornika, maksymalnej amplitudy sygnału audio jaka może zostać podana na wyjściowe wzmacniacze mocy. Drugi stopień regulacji jest dostępny z~zewnątrz urządzenia i~służy do ustawiania przez użytkownika poziomu głośności odbieranej stacji. 
				$ $\\
				
				Jako wyjściowy wzmacniacz mocy wykorzystano gotowy moduł z~układem PAM8403\textsuperscript{\cite{hw_amp}}, który steruje głośnikami o~mocy 3W i~impedancji 4$\Omega$. Moduł jest wyposażony w~wejście \lstinline|MUTE| wykorzystywane do całkowitego wyciszenia wzmacniacza w~sytuacji, w~której nie jest odbierany strumień danych ze stacji radiowej.
				
			\subsubsection{Raspberry Pi Pico W}
				Schemat ilustrujący sposób podłączenia modułu uruchomieniowego Raspberry Pi Pico~W z~resztą podzespołów odbiornika został przedstawiony na rysunku~\ref{3/hw_kicad_sch_picow}.
%			 	Składa się on głównie z~płytki rozwojowej Raspberry Pi Pico~W, która zawiera układ mikrokontrolera RP2040 oraz pamięć oprogramowania układowego typu FLASH. Znajduje się tam także moduł komunikacji bezprzewodowej Wi-Fi.
				\imgh{3/hw_kicad_sch_picow}{Schemat elektryczny radia: Raspberry Pi Pico~W}{0.65}
				
				Moduł czytnika kart SD podłączony jest do wyprowadzeń \lstinline|GPIO1| - \lstinline|GPIO6|.
				Interfejs \isqs{} toru audio podłączony został do wyprowadzeń \lstinline|GPIO7| - \lstinline|GPIO9|.
				Moduł wyświetlacza korzysta z~wyprowadzeń \lstinline|GPIO10| - \lstinline|GPIO15|. Klawiaturę sterującą podłączono do wyprowadzeń \lstinline|GPIO18| - \lstinline|GPIO22|. Wyjścia dzielników pomiarowych napięć sekcji zasilającej wprowadzono na wejścia przetwornika analogowo-cyfrowego: \lstinline|GPIO27| i~\lstinline|GPIO28|. Sterowanie wyciszeniem wzmacniacza mocy odbywa się przez \lstinline|GPIO26|.
				$ $\\
				
				Z~głównego modułu wyprowadzone zostało 5-pinowe złącze komunikacyjne. Znajduje się ono na tylnej ścianie obudowy i~zawiera sygnały:
				\begin{enumerate}
					\setlength{\itemsep}{0cm}
					\item \lstinline|SWCLK| -- linia zegarowa interfejsu SWD\textsuperscript{\cite{swd}},
					\item \lstinline|GND| -- masa układu,
					\item \lstinline|SWDIO| -- dwukierunkowa linia danych interfejsu SWD,
					\item \lstinline|RX| -- linia odbierania danych interfejsu \lstinline|UART0|,
					\item \lstinline|TX| -- linia nadawania danych interfejsu \lstinline|UART0|.
				\end{enumerate}
				
				Interfejs SWD służy do wgrywania oraz debugowania oprogramowania układowego.
				Pin \lstinline|TX| służy do wysyłania tekstowych informacji o~stanie urządzenia z~poziomu oprogramowania układowego.
				Pin \lstinline|RX| obecnie nie jest używany.
				
		\subsection{Płytka drukowana}
			Na podstawie wyżej opisanego schematu została zaprojektowana i~wykonana płytka drukowana.
			Zapewnia ona odpowiednią trwałość mechaniczna i~dobrą jakość połączeń elektrycznych pomiędzy poszczególnymi modułami wykorzystanymi do skonstruowania urządzenia.
			$ $\\

			Na rysunku~\ref{3/hw_kicad_pcb} przedstawiony został projekt mozaiki ścieżek i~rozmieszczenia elementów obwodu drukowanego. Zaznaczono na nim położenie dwóch najważniejszych sekcji. W~centrum znajduje się płytka Raspberry Pi Pico~W, a~w jej sąsiedztwie slot na karty SD oraz przyłącza do ekranu LCD (po lewej, 8-pinowe) i~przycisków (po prawej, 6-pinowe). Obok, na rysunku~\ref{3/hw_pcb}, przedstawiona została zmontowana płytka.
			
			\imghss{3/hw_kicad_pcb}{Projekt płytki}{3/hw_pcb}{Zmontowana, gotowa płytka (bez modułów)}
			
		\subsection{Obudowa}
			Po zwymiarowaniu wszystkich komponentów systemu i~płytki PCB powstał projekt obudowy przeznaczonej do wykonania w~technologii druku 3D.
			$ $\\
			
			Obudowa została zaprojektowana w~programie OpenSCAD\textsuperscript{\cite{hw_openscad}}. Wybór programu jest nieoczywisty ponieważ umożliwia on projektowanie 3D oparte o~prosty język skryptowy. Dzięki temu podejście do projektowania może tylko nieznacznie różnić się od pisania oprogramowania. Fragment kodu opisujący wygląd jednego z~elementów obudowy został przedstawiony na listingu~\ref{lst/3/buttons.scad}. Linie 3-4 generują siatkę 3x3 na której zostaną umieszczone przyciski/otwory na śruby. Operator trójargumentowy w~liniach 7-8 powoduje że w~rogach siatki otwory będą miały inną średnicę. Warto też zwrócić uwagę na wykorzystanie nazw zamiast wartości, co umożliwia łatwiejszą modyfikację.
			
			\lstfile{scad}{Kod generujący otwory na przyciski i~śruby montażowe}{lst/3/buttons.scad}
			\pagebreak
			
			Na rysunku~\ref{3/hw_scad_case} przedstawiono wizualizacja projektu całej obudowy. Poniżej, na rysunku~\ref{3/hw_case}, zaprezentowano efekt finalny, funkcjonalne radio w~obudowie.
			\imgh{3/hw_scad_case}{Wizualizacja projektu obudowy radia}{0.8}
			\imgh{3/hw_case}{Efekt końcowy, radio w~obudowie}{0.8}
			\pagebreak
	
	\section{Struktura oprogramowania układowego}
		Podczas procesu implementacji wymagań projektowych, wyodrębniane były kolejne funkcjonalne elementy ostatecznego programu.
		Na podstawie tego podziału stworzona została struktura modularna przedstawiona na rysunku \ref{3/PicoRadio-code-block-diagram}.
		\imgh{3/PicoRadio-code-block-diagram}{Schemat blokowy modułów projektu}{0.75}
		
		\subsection{Moduły własne}
			Większość kodu projektu zawiera katalog \lstinline|libs|. W~nim znajdują się poszczególne moduły (nazwane odpowiednimi nazwami podfolderów). Wszystkie wypisane w~tej sekcji zostały zaimplementowane samodzielnie, chyba, że zaznaczono inaczej. Dla przejrzystości pogrupowano moduły wg ich ról.
		
			\paragraph{Wejście}
				\begin{itemize}
					\item \lstinline|analog| -- dostęp do pomiaru poziomu naładowania baterii,
					\item \lstinline|buttons| -- funkcje do obsługi przycisków,
					\item \lstinline|sd| -- wykrywanie i~montowanie karty SD z~systemem plików FAT.
				\end{itemize}
			\pagebreak
			
			\paragraph{Wyjście}
				\begin{itemize}
					\item \lstinline|display| -- obsługa wyświetlacza:
					\begin{itemize}
						\item \lstinline|assets| -- artefakty; takie jak czcionki i~ikony,
						\item \lstinline|screens| -- widoki użytkownika,
						\item \lstinline|tft| -- niskopoziomowa obsługa ekranu ze sterownikiem ST7735s. Kod inicjalizacji i~przesyłania pikseli zaczerpnięty z~\textit{Adafruit-ST7735-Library}\textsuperscript{\cite{adafruit_st7735}}.
					\end{itemize}
				\end{itemize}
			
			\paragraph{Obsługa list}
				\begin{itemize}
					\item \lstinline|list| -- obsługa list jak i~ich ładowanie (wyszukiwanie/ulubione/listing plików na karcie SD).
				\end{itemize}
			
			\paragraph{Odtwarzanie}
				\begin{itemize}
					\item \lstinline|mcorefifo| -- funkcje pomocnicze do przesyłania komunikatów z~rdzenia 1~do 0,
					\item \lstinline|player| -- kontrola i~odtwarzanie plików/strumieni internetowych:
					\begin{itemize}
						\item \lstinline|decode| -- abstrakcje/implementacje dekodowania typów i~odczytywania źródeł danych,
						\item \lstinline|metadata| -- dekodowanie metadanych w~różnych formatach audio.
					\end{itemize}
				\end{itemize}
		
			\paragraph{Źródła danych}
				\begin{itemize}
					\item \lstinline|ds| -- interfejs \lstinline|DataSource| łączący różne źródła danych,
					\item \lstinline|httpc| -- klient protokołu HTTP,
					\item \lstinline|lfs| -- kod integrujący system plików LittleFS\textsuperscript{\cite{littlefs}}.
				\end{itemize}
			
			\paragraph{Wi-Fi}
				\begin{itemize}
					\item \lstinline|wifi| -- obsługa sieci bezprzewodowych.
				\end{itemize}
		
			\paragraph{Pliki pomocnicze}
				\begin{itemize}
					\item \lstinline|circularbuffer| -- bufor kołowy,
					\item \lstinline|static| -- statycznie alokowane bufory,
					\item \lstinline|util| -- funkcje pomocnicze,
					\item \lstinline|settings| -- wpisy na ekranie ustawień.
				\end{itemize}
	
		\subsection{Moduły zewnętrzne}
		W~folderze \lstinline|libs/external| umieszczone zostały moduły dostarczone z~zewnątrz:
		\begin{itemize}
			\item \lstinline|FreeRTOS-Kernel| -- jądro systemu czasu rzeczywistego FreeRTOS\textsuperscript{\cite{freertos_kernel}},
			\item \lstinline|helixmp3| -- biblioteka dekodująca format MP3\textsuperscript{\cite{helixmp3_repo}} oryginalnie napisana przez firmę \textit{RealNetworks}\textsuperscript{\cite{realnetworks}},
			\item \lstinline|littlefs| -- system plików LittleFS\textsuperscript{\cite{littlefs}},
			\item \lstinline|sd_lib| -- biblioteka FatFS\textsuperscript{\cite{fatfs}} wraz z~kodem dostępu do karty SD\textsuperscript{\cite{sdfs}}.
		\end{itemize}

	\section{Odtwarzanie dźwięku}
%		Zanim radio urosło do rangi pracy inżynierskiej, było ono moim prywatnym projektem. Chciałem poznać mikrokontroler RP2040. Poznać \lstinline|pico-sdk|. Po pierwszym kontakcie na \textit{Systemach Wbudowanych} środowisko stworzone wokół przystępnej cenowo płytki Raspberry Pi Pico wydawało się ciekawe i~interesujące. Mikrokontroler ten, bogato wyposażony, stanowił idealną podstawę do stworzenia czegoś, co zawsze chciałem zbudować: odtwarzacza dźwięku. Przy okazji, tworząc taki projekt mogłem zaznajomić się ze wszystkimi ważniejszymi peryferiami układu (wykorzystałem m.in. PWM, DMA, a~także później PIO).
%		
%		Szybko jednak okazało się, że odtwarzania dźwięku za pomocą wbudowanego układu PWM jest niewystarczające. Dźwięk był zbyt niskiej jakości aby nadawał się do odtwarzania muzyki. Wykorzystałem układ TDA1543 jako przetwornik cyfrowo-analogowy. Projekt odtwarzał tylko pliki WAV i~wyglądał tak, jak na rysunku \ref{3/pr_wav}.
%		
%		Wtedy też wybrałem ten temat pracy inżynierskiej. Lecz zanim mogłem kontynuować, należało szybko zorientować się, czy płytka poradzi sobie z~dekodowaniem formatu MP3 w~czasie rzeczywistym. Dzięki bibliotece \lstinline|helixmp3| używającej arytmetyki stałoprzecinkowej, było to możliwe.
%		
%		\imgcs{3/pr_wav}{Projekt we wczesnych fazach rozwoju}{0.45}
		
%		\subsection{Model przetwarzania dźwięku}
		
%			\imgcs{3/pipeline}{Koncepcja wielowątkowego dekodowania}{0.8}
			
		Odtwarzanie dźwięku odbywa się za pomocą zewnętrznego przetwornika DAC. Mikrokontroler, wykorzystując DMA, wysyła dźwięk w~formacie PCM protokołem \isqs{} do specjalizowanego układu przetwornika audio. Jeden z~rdzeni procesora wykorzystywany jest wyłącznie do dekompresji formatów audio.
		
	%	TODO opisać tutaj co robi player.cpp (zmienna dec, funkcje publcizne)
		
		\subsection{Komunikacja z~przetwornikiem}
			Komunikacja ze sprzętową częścią audio odbywa się poprzez podsystem DMA i~bloki programowalne PIO.
			W~programie utworzony został bufor nieskompresowanego dźwięku PCM \lstinline|audio_pcm|. Ma on konfigurowalny rozmiar wyrażony w~jednostkach 32-bitowych (2 kanały * 16-bitowa próbka). Przekazywany jest on do systemu DMA, który następnie transmituje jego zawartość poprzez PIO zaprogramowane do symulowania interfejsu \isqs{}. Oba etapy transmisji zostały opisane poniżej.
		
		\subsection{PIO}
			RP2040 posiada 8~programowalnych maszyn stanów, zgrupowanych w~2~bloki.
			Maszyny te można wykorzystać do symulacji szerokiej gamy interfejsów komunikacyjnych.
			Poprzez ich zastosowanie, możliwe jest m.in. utworzenie dodatkowego bloku komunikacji UART czy SPI.
			Jednak największy potencjał PIO wykazuje gdy w~układzie brak jest obsługi niezbędnego w~projekcie interfejsu, np. CAN czy \isqs{}.
			Schemat budowy bloku PIO przedstawiono na rysunku~\ref{3/pio_block}.
			\imgh{3/pio_block}{Pojedynczy blok PIO (dokumentacja RP2040\textsuperscript{\cite{pico_pdf}}, strona 309)}{0.75}
			
			W~ramach jednego bloku maszyny współdzielą pamięć 32 instrukcji. Może wydawać się to mało, lecz instrukcje te są bardzo skondensowane. Język do programowania PIO to własnościowy język z~rodziny assembler. Do środowiska \lstinline|pico-sdk| dołączony jest assembler \lstinline|picoasm|. Zestaw instrukcji przedstawiono na rysunku \ref{3/pio_instr}.
			\imgh{3/pio_instr}{Zestaw instrukcji PIO (dokumentacja RP2040, strona 320)}{0.7}
			
			Jak wspomniano wcześniej, instrukcje są skondensowane, to znaczy mogą wykonywać wiele operacji na raz. Na przykład opóźniać wykonanie programu lub ustawiać osobny rodzaj wyjść typu \sset{}.
			Instrukcja skoku obsługuje także dekrementację zmiennej w~warunku. Same maszyny mogą mieć ustawiony dowolny adres wejścia i~zapętlenia programu. Oszczędza to kilka instrukcji.
			
			\paragraph{Wyjścia typu \sset{}}
				Jest to specjalny rodzaj wyjść, które mogą być ustawiane równocześnie z~wykonywaniem innych instrukcji. Są szczególnie przydatne przy ustawianiu linii zegarowych i~pozostałych linii kontrolnych. Minusem wykorzystania ustawiania wyjść \sset{} jest konieczność poświęcenia możliwości opóźniania programu. W~liście instrukcji pole \sset{} jest współdzielone z~polem \lstinline|delay|. Jednak zaimplementowany program nie używa opóźnień.
			$ $\\
			
			Dzięki tym maszynom, możliwe jest dodanie do mikrokontrolera dowolnego protokołu transmisji danych bez zbytniego obciążenia procesora jego pełną symulacją.
			W~ramach pracy został stworzony blok obsługi interfejsu \isqs{}, który serializuje dane dostarczane przez DMA i~wysyła do układu DAC. Kod tego bloku przedstawiony jest na listingu \ref{lst/3/i2s.pio}.
			$ $\\
			
			Przebieg czasowy protokołu \isqs{} przedstawia rysunek~\ref{3/pio_i2s_docs}.
			Został on zaczerpnięty z~dokumentacji układu TDA1543 (przetwornik cyfrowo-analogowy zastosowany we wczesnych prototypach).
			\imgh{3/pio_i2s_docs}{Przebieg czasowy protokołu \isqs{} (dokumentacja TDA1543\textsuperscript{\cite{tda_pdf}}, strona 9)}{0.9}
			
%			Jak widać zmiana kanału (kodowanego poziomem linii \lstinline|WS|)  musi nastąpić przed wysłaniem ostatniego bitu poprzedniego kanału (\lstinline|LSB|, czyli \textit{Least Significant Bit}, najmniej znaczący bit kanału lewego). Z~tego powodu program wydłużył się do 8~instrukcji. Niektóre układy obsługują protokół \textit{left-just}, który nie posiada tego ograniczenia, jednak z~uwagi na niewystarczającą dokumentacje i~małe wsparcie przez układy, wybrany został klasyczny \isqs{}.
			
			\lstfile{pio}{Kod programu \isqs{}}{lst/3/i2s.pio}
			
			\noindent
			Kod bloku PIO z~listingu~\ref{lst/3/i2s.pio} odpowiada za realizacje transmisji w~standardzie \isqs{} ilustrowanej przebiegiem pokazanym na rys~\ref{3/pio_i2s_docs} i~działa następująco:
			\begin{itemize}
				\item Linia nr 3~ustawia ile bitów jest typu \lstinline|side-set|. Bity \lstinline|side-set| są wykorzystywane odpowiednio do: ustawienia linii \lstinline|WS| i~linii \lstinline|BCK|.
				
				\item Punktem wejścia programu jest linia nr 18. Ustawia ona liczbę powtórzeń i~inicjalizuje linię \lstinline|WS| na 1~(kanał prawy) oraz linię \lstinline|BCK| na 1. Odwrócenie kolejności kanałów związane jest z~kolejnością bajtów typu \textit{little-endian}). Następnie program automatycznie zapętla się do pierwszej instrukcji.
				
				\item Instrukcja \lstinline|out| powoduje wystawienie na linię \lstinline|DATA| jednego bitu z~rejestru przesuwnego danych. Przy pomocy funkcjonalności \sset{} przełącza linię zegarową w~stan niski.
				
				\item Instrukcja \lstinline|jmp| najpierw sprawdza wartość rejestru \lstinline|X|. Jeżeli jest ona niezerowa to dekrementuje ją i~skacze do etykiety \lstinline|right|. Pętla ta wykonuje się 15 razy.
				
				\item W~linii nr 9~następuje transmisja najmniej znaczącego bitu kanału prawego. Zgodnie z~dokumentacją musimy już zmienić stan linii \lstinline|WS| na 0~(kanał lewy). Następnie rejestr \lstinline|X| jest ponownie inicjalizowany na wartość 14.
				
				\item Linie 13-18 powtarzają transmisje zgodnie z~opisem powyżej, ale tym razem dla kanału lewego.
			\end{itemize}
			
			Instrukcja \lstinline|out| czeka na dane w~buforze automatycznie.
		
		\subsection{DMA}
			Użyty mikrokontroler posiada 12~kanałów DMA.
			Pojedynczy kanał może służyć m.in. do transferów pomiędzy pamięcią a~wybranym układem peryferyjnym, np. blokiem PIO.
			Poprzez zastosowanie tego mechanizmu procesor zostaje odciążony i~może wykonywać inne zadania.
			$ $\\
			
			\newcommand{\rdmabuf}[1]{\ref{3/PicoRadio-dma-buf}#1}
			\newcommand{\dmacite}{\textsuperscript{\cite{dma_dbl_buf1} \cite{dma_dbl_buf2}}}
			
			Aby zachować płynność odtwarzania, wykorzystano podwójne buforowanie\dmacite{}. Na rysunku~\rdmabuf{} przedstawiono schematy blokowe ilustrujące jego działania.
			
			\imgh{3/PicoRadio-dma-buf}{Schemat blokowy podwójnego buforowania DMA\dmacite{}}{1}
			
			Użyte zostały dwa osobne kanały DMA. Każdy z~nich został skonfigurowany aby przesyłał połowę bufora nieskompresowanego dźwięku PCM.
			$ $\\
			
			Na rysunku~\rdmabuf{a} przedstawiono proces wykonywany przed rozpoczęciem odtwarzania.
			Bufor wypełniany jest w~całości.
			Po rozpoczęciu odtwarzania podsystem DMA kopiuje sukcesywne próbki z~bufora do maszyny PIO. Podczas pierwszego transferu procesor nie musi nic robić, ponieważ załadował on na początku cały bufor. Sytuację tę ukazuje rysunek~\rdmabuf{b}.
			$ $\\
			
			W~momencie, gdy transfer pierwszej części bufora się zakończy, automatycznie, bez ingerencji procesora, uruchamiany jest drugi kanał DMA. Jednocześnie wywoływane jest przerwanie, które informuje procesor, że należy załadować więcej danych do pierwszej połowy bufora. Cały proces został zilustrowany na rysunku~\rdmabuf{c}.
			Ukończenie przesyłania drugiej części bufora przebiega analogicznie. Uruchamiany jest pierwszy kanał DMA i~procesor uzupełnia drugą część bufora. Sytuację przedstawia rysunek~\rdmabuf{d}.
			$ $\\
		
			Stany przedstawione na rysunkach~\rdmabuf{c} oraz \rdmabuf{d} powtarzane są dopóki odtwarzanie nie zostanie zakończone. W~ten sposób przebiega ono w~sposób płynny. Kanały uruchamiane są w~pętli, z~minimalnym opóźnieniem, na poziomie sprzętowym. 			
%			Wykonanie przerwania procesora trwa relatywnie długo, co mogłoby powodować utratę płynności w~odtwarzaniu dźwięku.
		
		\subsection{Dekompresja formatów}
			\label{sec:decode}
			
			Mikrokontroler RP2040 zawiera 2~rdzenie Cortex~M0+. Wykorzystanie obu jest niezbędne dla zapewnienia odpowiedniej wydajności pobierania i~przetwarzania danych zawierających dźwięk, zwłaszcza gdy jest on zakodowany w~formacie MP3.
			$ $\\
			
			Rysunek \ref{3/PicoRadio-sound-decoding} przedstawia w~jaki sposób zorganizowane zostało dekodowanie i~odtwarzanie dźwięku.
			Rdzeń 0~odpowiada za dostarczenie (pobranie z~serwera stacji lub z~pliku na karcie SD) surowych zakodowanych danych oraz obsługę (poprzez DMA) regularnego wysyłania rozkodowanych danych audio do przetwornika cyfrowo-analogowego.
			Rdzeń 1~odpowiada za dekodowanie danych z~formatu wejściowego do postaci PCM.
			\imgh{3/PicoRadio-sound-decoding}{Architektura dekodowania dźwięku}{0.95}
			
			Struktura taka umożliwia łatwą rozbudowę oprogramowania dekodującego o~obsługę innych formatów danych. Dodając obsługę nowego formatu należy mieć na uwadze, że proces dekodowania danych z~tego formatu musi się dać zrealizować w~czasie rzeczywistym na pojedynczym rdzeniu Cortex M0+.
			
		%	\paragraph{Flagi przerwań}
		%	Jak wcześniej wspomniano, przerwania generowane przez kanały DMA ustawiają dwie flagi \lstinline|a_done_irq| lub \lstinline|b_done_irq|. Poprzez wspólną pamięć są one odczytywane przez rdzeń 1, a~ten dekoduje dane i~uzupełn
		
			\paragraph{Część Decode}
				Komponenty z~tej rodziny odpowiedzialne są za dostarczanie nowych danych surowych (w kodzie \lstinline|raw_buf| (\textit{raw buffer}) lub \lstinline|cbuf| (\textit{content buffer})). Klasa bazowa \lstinline|DecodeBase| odpowiada za wspólne elementy obu źródeł danych. Obsługuje ona m.in. zunifikowane funkcje do wysyłania komunikatów o~zakończeniu dekodowania, zużyciu danych z~bufora, bądź błędach. Umożliwione zostało także odczytywanie statystyk i~metadanych. Poszczególne klasy podrzędne skupiają się na dostarczaniu danych z~różnych źródeł. \lstinline|DecodeFile| odpowiada za odczyt plików z~karty SD, natomiast \lstinline|DecodeStream| odbiera dane ze strumienia internetowego.
			
			\paragraph{Część Format}
				Powstały interfejs \lstinline|Format| służy do abstrakcji dekodowania danych surowych. Jego główną metodą jest \lstinline|decode_up_to_n|, która dekoduje co najwyżej \lstinline|n|~jednostek danych. Aby dowiedzieć się ile jednostek należy zdekodować istnieją funkcje \lstinline|units_to_decode_whole| i~\lstinline|units_to_decode_half|, zwracające odpowiednio ile jednostek zapełni cały bufor \lstinline|audio_pcm| a~ile pół bufora (taka liczba jest dekodowana po otrzymaniu przerwania od DMA, którego jeden kanał transferuje pół bufora). W~przypadku formatu WAV jednostki to po prostu bajty, a~gdy mówimy o~MP3 będą to ramki. Podklasy implementują specjalizowane funkcje do dekodowania poszczególnych formatów.
			
			\paragraph{Bufor FIFO}
				RP2040 posiada wbudowane dwa bufory FIFO do komunikacji między rdzeniami. Jeden z~rdzenia 0~do rdzenia 1, drugi w~przeciwnym kierunku. Wykorzystany został bufor w~kierunku $1 \rightarrow 0$ do przekazywania informacji o~stanie dekodowania. Zaimplementowany został moduł \lstinline|mcorefifo| do wysyłania abstrakcyjnych wiadomości. Wykorzystuje on przerwanie SIO do odbioru komunikatów, a~następnie przekazuje je do zarejestrowanego odbiorcy. Jednym z~takich odbiorców jest moduł odtwarzacza, który w~pliku \lstinline|decodebase.cpp| odbiera komunikaty o~identyfikatorze \lstinline|PLAYER| w~funkcji \lstinline|player_msg()| i~przekazuje do swojej głównej pętli w~funkcji \lstinline|play()|.
		
		\subsubsection{Biblioteka \textit{helixmp3}}
			Do dekodowania formatu MP3 wykorzystana została biblioteka \lstinline|helixmp3|.
%			Opiera się ona na arytmetyce stałoprzecinkowej.
			Zaimplementowany w~niej dekoder formatu MP3 wykorzystuje arytmetykę stałoprzecinkową co pozwala na efektywne jego wykorzystanie na maszynach nie mających sprzętowego wsparcia obliczeń zmiennoprzecinkowych.
			Oryginalnie napisana przez \textit{RealNetworks}\textsuperscript{\cite{realnetworks}} w~2003~roku, znaleziona została w~Internecie pod postacią repozytoria GitHub\textsuperscript{\cite{helixmp3_repo}}. Do biblioteki dodana została obsługa RP2040 jak również integracja ze środowiskiem CMake.
%			Wspomniane repozytorium GitHub to pierwszy wynik w~wyszukiwarce Google (fraza \textit{mp3 decoding library fixed point}), dlatego
			Poczynione ulepszenia, zostały zgłoszone autorowi repozytorium\textsuperscript{\cite{helixmp3_pr}}. Umożliwiona została także statyczna alokacja buforów na dane dekodera.
			$ $\\
			
			Do biblioteki dekodującej odwołuje się tylko \lstinline|FormatMP3|, który funkcją \lstinline|decode_up_to_n|, dekoduje co najwyżej \lstinline|n| ramek MP3. Zwracana jest liczba zdekodowanych ramek. Jeżeli zdekoduje ich mniej, to znaczy, że odtwarzanie zakończyło się pomyślnie (np. koniec pliku). Jeżeli zwrócona zostanie wartość mniejsza od zera, oznacza to błąd. Jeżeli danych w~buforze brakuje (np. niestabilna komunikacja sieciowa), rdzeń 1~czeka na dane w~nieskończoność. Użytkownik jednak w~każdej chwili może przerwać oczekiwanie.
		
		\subsection{Rozpoczynanie odtwarzania}
			Użytkownik rozpoczyna odtwarzanie poprzez wybranie pozycji z~listy ulubionych lub wyszukanie stacji/wybór pliku na karcie SD.
			Rysunek~\ref{3/PicoRadio-start} przedstawia schemat blokowy procesu rozpoczęcia odtwarzania.
			$ $\\
			
			Dowolny widok może zainicjować ekran odtwarzania \lstinline|ScPlay|.
			Jednym z~parametrów inicjalizacyjnych jest ścieżka dostępu. Gdy nastąpi zmiana aktywnego ekranu, wywoływana jest funkcja \lstinline|player_start()|. Więcej na temat widoków i~zarządzania ekranem napisano w~sekcji~\ref{sec:screen}.
			
			\paragraph{Zadania systemu FreeRTOS}
				Funkcja \lstinline|player_start()| tworzy dwa osobne zadania w~systemie operacyjnym, a~sama kończy działanie od razu. Pierwsze z~nich, \lstinline|player|, odpowiedzialne jest za uruchomienie procesu odtwarzania i~przetwarzanie komunikatów (np. odnośnie konsumpcji bajtów z~bufora czy wystąpienia błędów). Ładuje ono także dane do bufora z~pliku. Drugie, pomocnicze zadanie \lstinline|player stat| służy do okresowego aktualizowania stanu odtwarzania na ekranie.
			
			\paragraph{Wykrywanie formatu}
				Format audio wykrywany jest na podstawie ścieżki do zasobu. Jeżeli ta zaczyna się od \lstinline|http|, zostanie użyta klasa \lstinline|DecodeStream| i~\lstinline|FormatMP3|. Jeżeli początek nie dopasuje się do wzorca, przyjmowane jest odtwarzanie z~pliku (klasa \lstinline|DecodeFile|) i~na podstawie rozszerzenia pliku definiowane jest użycie klasy \lstinline|FormatWAV| lub \lstinline|FormatMP3|.
				
			\imgh{3/PicoRadio-start}{Schemat procesu rozpoczynania odtwarzania}{0.7}
				
			\paragraph{Ograniczenia detekcji plików}
				Definicja formatu audio przebiega na samym początku co jest poważnym ograniczeniem i~nie pozwala na zmianę formatu np. po odczytaniu nagłówka \lstinline|Content-type|. Jednak w~obecnej formie projekt obsługuje tylko strumieniowanie plików MP3. Gdyby została dodana obsługa stacji AAC/AAC+ należałoby poprawić ten fragment kodu. Dodatkowo wtedy detekcja formatu plików lokalnych mogłaby zostać przeniesiona do \lstinline|DecodeFile|. Oprócz rozszerzeń, możliwa byłaby detekcja formatów plików na podstawie ich nagłówków. Pliki rozpoczynające się od \lstinline|http| zostaną potraktowane jako strumienie audio. Należy rozważyć użycie prefiksu \lstinline|file://| dla plików lokalnych.
			
			\paragraph{Możliwe błędy}
				Jeden z~możliwych błędów występuje, gdy nie zostanie dopasowany żaden znany format audio. Należy wtedy zweryfikować czy pliki posiadają odpowiednie rozszerzenia, a~strumienie sieciowe prefiks \lstinline|http|. Inny błąd oznacza że nie udało się rozpocząć odtwarzania, tzn. plik nie istnieje/nie można go odczytać lub serwer strumieniujący nie odpowiada lub też nie udało się określić jego adresu. Przedstawione poniżej zakańczanie odtwarzania nie uwzględnia tych błędów, ponieważ z~punktu widzenia programu jeszcze się ono nie rozpoczęło.
		
		\subsection{Zatrzymywanie odtwarzania}
			Zakończenie przetwarzania audio jest nietrywialnym procesem. Sygnał do zatrzymania może pochodzić z~trzech różnych źródeł. Na rysunku~\ref{3/PicoRadio-stop} przedstawione zostały kolejne etapy obsługi tych sygnałów. Natomiast poniżej opisano skrótowo jak działa zgłaszanie poszczególnych kategorii zdarzeń. 
			
			\imgh{3/PicoRadio-stop}{Schemat obsługi żądań zakończenia odtwarzania}{1}
			
			Parametrem funkcji \lstinline|DecodeBase::notify_playback_end(bool error)| jest flaga, która mówi, czy odtwarzanie zakończyło się błędem. Jeżeli jest ona ustawiona, to program wyświetli ekran błędu.
			
			\paragraph{Użytkownik}
				Użytkownik może zakończyć odtwarzanie poprzez naciśnięcie przycisku \textit{Wstecz} na ekranie odtwarzania. Podsystem wyświetlacza wywołuje wtedy funkcje \lstinline|player_stop()|, która rozpoczyna proces zakończenia odtwarzania i~czeka aż rzeczywiście się ono zakończy. Ustawiana jest flaga \lstinline|abort|, która powoduje, że wszystkie implementacje \lstinline|decode_up_to_n(int n)|, kończą dekodowanie natychmiast. Dekodowane jest mniej ramek niż zadane \lstinline|n|, co wykrywane jest przez program jako zakończenie odtwarzania.
			
			\paragraph{Koniec pliku}
				Gdy funkcja \lstinline|load_buffer()| w~\lstinline|DecodeFile| napotka koniec pliku powiadamia o~tym fakcie klasę nadrzędną wywołując \lstinline|notify_eof()|. Nie używa ona bezpośrednio funkcji \lstinline|notify_playback_end()|, ponieważ przy zakończeniu odtwarzania bufor może zawierać jeszcze do 4~sekund dźwięku (bufor 64kB, format MP3 128kbps). Dopiero gdy funkcja oczekiwania na dane wykryje że jest ich za mało, a~w pliku nie ma więcej bajtów do odczytania, odtwarzanie jest zakańczane. Dzieje się to w~analogiczny sposób jak w~przypadku zakończenia go przez użytkownika: dekodowane jest mniej ramek audio.
			
			\paragraph{Błąd}
				Błędy modułu odtwarzacza mogą być dwojakie. Błąd sieciowy występuje przy odtwarzaniu strumieni internetowych, np. radia. Biblioteka \textit{LWIP} zgłasza błąd, który poprzez \lstinline|notify_playback_end()| powoduje natychmiastowe zatrzymanie odtwarzania i~wyświetlenie komunikatu o~błędzie (flaga \lstinline|error| ustawiona na \lstinline|true|). Błąd odczytu z~pliku występuje, gdy funkcja uzupełniająca bufor o~dane z~pliku zakończy się niepowodzeniem. Oznaczać to może na przykład że użytkownik usunął kartę SD z~urządzenia lub wystąpił błąd komunikacji z~nią. Analogicznie jak w~przypadku błędu sieci odtwarzanie jest zakańczane natychmiast.
			
		\subsection{Metadane}
		Jednym z~wymagań projektu było wyświetlanie informacji na temat aktualnie odtwarzanego utworu. Należało więc co najmniej zdekodować metadane wysyłane przez stacje radiowe. Dodatkowo, zostało zaimplementowane odczytywanie metadanych z~plików MP3.
		
		\subsubsection{Stacje radiowe -- ICY}
			Szeroko stosowanym formatem przesyłania danych o~utworach w~strumieniach radiowych jest format ICY\textsuperscript{\cite{icy_spec}}. Klient może zażądać od serwera metadanych wysyłając do niego nagłówek \lstinline|Icy-Metadata: 1|. Serwer, jeżeli zaakceptował żądanie, odsyła nagłówek \lstinline|icy-metaint: <n>|. Jego wartość oznacza co ile bajtów strumienia będą wysyłane ramki z~metadanymi. Zwykle \lstinline|<n>| jest równe 16000, co oznacza, że co każde odebrane 16000~bajtów wystąpią metadane. Jest to 1~sekunda dźwięku MP3 128kbps, czyli nazwy utworów będą aktualizowane raz na sekundę. Występuje też wartość 1000, co oznacza 16~aktualizacji w~ciągu sekundy. Inne wartości są możliwe, ale nie są spotykane w~praktyce.
			$ $\\
			
%			\paragraph{Abstrakcyjne położenie ramek ICY}
			Warto zaznaczyć, że miejsce wstawienia metadanych w~żaden sposób nie uwzględnia formatu audio. Nie możemy zakładać że nie podzielą one jednej ramki MP3 na dwie części. Dla dźwięku MP3 128kbps ramka ma rozmiar 418~bajtów. Przy wartości \lstinline|icy-metaint| równej 16000, dane dźwiękowe będą podzielone co ok. $38.28$ ramki. Z~tego powodu konieczne stało się obsługiwanie ramek ICY w~momencie zapisu do bufora, zamiast przy jego odczycie. Odczyt obsługuje odpowiednia instancja \lstinline|Format|, która nie musi obsługiwać odczytywania ramek bajt po bajcie. Na przykład \lstinline|FormatMP3| nie obsługuje dekodowania ramek w~częściach. Ramka MP3 musi znaleźć się w ciągłym obszarze pamięci, nie może być rozdzielona ramką ICY.
			$ $\\
			
			Funkcja \lstinline|DecodeStream::play()|, jeżeli dostępny jest nagłówek \lstinline|icy-metaint|, inicjalizuje odbierane danych metodą \lstinline|ICY::start()|. Następnie rejestruje odwołanie, które jest wywoływane po każdym zapisie do bufora oraz przetwarza już odebrane dane. W~trakcie tych czynności blokowany jest kontekst biblioteki sieciowej, aby w~trakcie przetwarzania danych nie zostały odebrane żadne nowe dane, a~po odblokowaniu każde następne były wychwycone przez funkcję \lstinline|cbuf_write_cb()|.
			$ $\\
			
			W~każdym z~tych scenariuszy (wstępne i~późniejsze przetwarzanie) wywoływana jest w~pętli funkcja \lstinline|ICY::read()|. Pętla jest potrzebna, ponieważ w~porcji odebranych danych, szczególnie dla niskich wartości \lstinline|icy-metaint|, może wystąpić więcej niż jedna ramka ICY.
			
			\paragraph{Format danych ICY}
				Nagłówek metadanych ICY składa się z~jednego bajtu kodującego rozmiar danych (bez nagłówka) podzielony przez 16. Dane dopełniane są zerami do podanego rozmiaru. Ramka ICY została przedstawiona na rysunku~\ref{3/PicoRadio-icy}.
				
				\imgh{3/PicoRadio-icy}{Ramka MP3 rozdzielona ramką ICY}{0.9}
				
				\noindent
				Wewnątrz ramki kolejne pola oddzielone są od siebie średnikami, a~same pola są w~formacie \lstinline|klucz='wartosc'|. Dekoder zaimplementowany w~projekcie obsługuje tylko klucz \lstinline|StreamTitle|, którego wartość to zazwyczaj tytuł utworu i~jego wykonawca.
				$ $\\
				
			Wywołanie \lstinline|ICY::read()| kopiuje ramkę ICY z~bufora audio do bufora tymczasowego, a~następnie przesuwa dane za ramką wstecz o~rozmiar ramki. Operację można porównać do wycinania tekstu z~dokumentu tekstowego. W~ten sposób przywracamy ciągłość ramce MP3. Dane w~buforze tymczasowym nie są interpretowane aby zaoszczędzić cenny czas. Nalezy pamiętać, że cała ta operacja odbywa się przy zablokowanej komunikacji sieciowej. Dane analizowane są dopiero gdy zadanie aktualizacji statystyk odtwarzania \lstinline|player stat| wywoła \lstinline|DecodeBase::get_meta_str()|, która w~implementacji \lstinline|DecodeStream| odwołuje się do metody \lstinline|ICY::get_stream_title()|. Funkcja ta blokuje mutex bufora tymczasowego i~kopiuje dane do swojego bufora na stosie. Dopiero wtedy rozpoczyna potencjalnie kosztowną czasowo interpretację danych.
		
		\subsubsection{Lokalne pliki MP3 -- ID3v2}
			Tagowanie MP3 obsługiwane jest zupełnie inaczej od ramek ICY. Tagi ID3v2 mogą występować na początku plików MP3. Sygnalizuje to fakt, że pliki te rozpoczynają się 3~bajtami reprezentującymi ciąg znaków \lstinline|ID3|. Funkcje do obsługi tego typu tagowań zostały zaimplementowane w~klasach pochodnych \lstinline|Format|. Wykorzystana została już wcześniej istniejąca funkcja \lstinline|Format::decode_header()|. Obsługiwała ona dekodowanie nagłówków WAVE. Wywoływana jest ona po pierwszym załadowaniu bufora, tylko raz na dany plik/strumień. Implementacja \lstinline|FormatMP3| wywołuje wtedy funkcję \lstinline|ID3::try_parse()|.
			
			\paragraph{Standard ID3v2}
			Ramki ID3 w~wersji drugiej mają znacznie bardziej rozbudowane kodowanie od np. ramek ICY czy starszego formatu ID3v1. Dokumentacja oryginalnie dostępna była pod adresem \href{id3.org}{id3.org} lecz w~trakcie pisania pracy była ona niedostępna. Skorzystano z~alternatywnej strony\textsuperscript{\cite{id3_spec_mutagen}}.
			
			\paragraph{Format ramek}
			Budowa formatu ID3v2 została przedstawiona na rysunku~\ref{3/PicoRadio-id3}. Podstawową jednostką budowy są tagi. Może być ich w~pliku dowolna liczba. Są one opakowane dodatkowymi informacjami, takimi jak np. nagłówki. Z~głównego nagłówka pobierany jest rozmiar i~flaga czy nagłówek rozszerzony jest obecny. Jeżeli tak, to jest on ignorowany. Padding i~stopka również są ignorowane.
			
			\imgh{3/PicoRadio-id3}{Format danych ID3v2}{0.8}
			
			Parser rozpoznaje ramki o~kodach \lstinline|TPE1|(artysta) i~\lstinline|TIT2|(tytuł). Są to ramki tekstowe, dlatego wyszczególniono taki format ramek na rysunku. Pierwszy bajt oznacza jakie kodowanie wykorzystano do zapisania tekstu. Możliwe formaty to ISO-8859-1, UTF-16, UTF-16BE i~UTF-8. Wszystkie zostały zaimplementowane i~przetestowane.
	
	\section{Bufor kołowy}
		\label{sec:circular_buffer}
		Małym, lecz kluczowym modułem w~projekcie jest bufor kołowy \lstinline|CircularBuffer|. Używany jest on przede wszystkim przy odtwarzaniu dźwięku. Schematyczne wykorzystanie przedstawiono na rysunku~\ref{3/PicoRadio-buffer}. Poniżej zostały opisane najważniejsze jego funkcje.
		
		\imgh{3/PicoRadio-buffer}{Schemat wykorzystania bufora kołowego}{0.8}
		
		\paragraph{Zapis i~odczyt danych przez różne wątki}
			Dzięki zastosowaniu osobnych liczników do zapisu i~odczytu danych, nie występuje problem wyścigu. Każda operacja może być wykonywana niezależnie przez co najwyżej jeden wątek.
			
		\paragraph{Funkcje zgłaszające}
			Moduł umożliwia zarejestrowanie funkcji, które zostaną wywołane po odczytaniu lub zapisaniu danych. Wykorzystywane są np. do potwierdzenia odbioru w~komunikacji sieciowej lub do wycięcia metadanych ICY.
		
		\paragraph{Sprawdzanie stanu zapełnienia i~czekanie}
			Udostępniony jest szereg funkcji określających stan bufora. Można odczytać pozycje odczytu/zapisu bezpośrednio, lecz poza procesem tworzenia oprogramowania, nie jest to przydatna metryka, ponieważ bufor "zapętla się". Istotniejszą metryką jest natomiast to, ile danych zostało zapisane do bufora a~ile odczytane od początku komunikacji. Dzięki tym danym możliwe jest jednoznaczne określenie, ile danych jest gotowych do odczytania w~danej chwili, oraz ile jest to procent całkowitej pojemności bufora.
		
		\paragraph{Zapętlanie}
			Najważniejszą funkcją bufora jest jego zapętlanie. Problem pojawił się w chwili implementacji dekodera MP3, ponieważ rozmiar bufora nie dzielił się całkowicie przez długość ramki. To znaczy, ostatnia ramka w~buforze była nieciągła. Pierwsza część mogła znaleźć się na końcu bufora, a~druga na jego początku. Problem rozwiązano poprzez dołożenie przed początkiem kilku kilobajtów pamięci, do której kopiowane są nieciągłe dane. Wskaźnik odczytu staje się wtedy ujemny. Poszczególne sekcje bufora zostały również pokazane na rysunku~\ref{3/PicoRadio-buffer}. Część normalna \lstinline|buffer| rozpoczyna się zerowym wskaźnikiem odczytu \lstinline|read_at|. Część ukryta \lstinline|buffer_hidden| znajduje się bezpośrednio przed nią.
		
		\paragraph{Wycinanie zapisanych danych}
			Implementacja wycinania stała się konieczna w~momencie kiedy w~środku niepodzielnej jednostki przy odczycie (np. ramki MP3) pojawiają się metadane jak np. ICY. Trzeba je wtedy przetworzyć i usunąć, zapełniając powstałą lukę danymi występującymi dalej w~buforze.
	
	\section{Klient protokołu HTTP}
%		Potrzeba stworzenia własnej implementacji klienta HTTP stała się konieczna, aby zintegrować go z~buforem kołowym i~biblioteką sieciową.
		W~ramach projektu powstała implementacja klienta HTTP. Zintegrowany on został z~buforem kołowym i~biblioteką sieciową.
		Klient obsługuje przekierowania i~podstawowe nagłówki. Wystarczająco, aby z~powodzeniem pobierać wyniki wyszukiwań stacji oraz odtwarzać je. Implementacja jest dwuczęściowa. Nadrzędny moduł \lstinline|HttpClient| obsługuje wysokopoziomowy standard HTTP. Zajmuje się komponowaniem żądań, ich nagłówków, a~także przetwarzaniem odpowiedzi. Podrzędny moduł obsługuje niskopoziomową komunikację sieciową. Taka budowa modułowa pozwoliła na testowanie klienta w~środowisku PC ze stosowną implementacją modułu podrzędnego. Urządzenie zawiera moduł \lstinline|HttpClientPico|, który współpracuje z~zasobami dostępnymi na mikrokontrolerze.
		$ $\\
		
		Obsługiwane nagłówki są ograniczone z~uwagi na chęć statycznej alokacji pamięci. Żądanie wysyłane jest z~nagłówkami:
		\begin{itemize}
			\item \lstinline|Host| o~wartości hosta z~którego pobierane są dane,
			\item \lstinline|User-agent: PicoRadio/0.1|,
			\item \lstinline|Icy-MetaData| o~wartości \lstinline|1|, jeżeli program ustawił odpowiednią flagę.
		\end{itemize}
		Natomiast przetwarzane są następujące nagłówki odpowiedzi:
		\begin{itemize}
			\item \lstinline|Location| -- wykorzystywane przy przekierowaniach,
			\item \lstinline|Content-Type| -- typ treści, używane głownie przy przetwarzaniu wyników wyszukiwania,
			\item \lstinline|Content-Length| -- długość treści, j.w.,
			\item \lstinline|Icy-Metaint| -- używane przy przetwarzaniu metadanych ICY.
		\end{itemize}
		
		Niestety brak jest obsługi szyfrowania. Środowisko \lstinline|pico-sdk| zawiera implementację TLS, lecz nie została ona wykorzystana z~uwagi na prawdopodobnie niewystarczającą moc obliczeniową mikrokontrolera. Wiele stacji transmituje z~użyciem protokołu HTTP, tylko niewielka część z~nich wymuszała przekierowania na szyfrowaną transmisję. Klient musi je obsługiwać ponieważ wiele stacji wykorzystuje je do geolokalizacji serwera transmitującego. W~konfiguracji zawarta została opcja \lstinline|HTTP_MAX_REDIRECTS| (domyślnie 2), która mówi, ile razy maksymalnie przekierowań chcemy obsłużyć. Zwykle jeżeli jest to więcej niż jedno, serwer próbuje wymusić transmisję szyfrowaną.
		
	\section{Ładowanie i~przechowywanie danych}
		Główną jednostką informacji w~projekcie jest stacja radiowa. Na potrzeby unifikacji dostępu do danych o~stacjach powstał interfejs \lstinline|DataSource|. Zawiera on dwie metody:
		\begin{itemize}
			\item \lstinline|read_char()| -- odczytuje znak,
			\item \lstinline|more_content()| -- zwraca czy w~źródle jest więcej danych.
		\end{itemize}
		Nad metodą czytającą znak po znaku zaimplementowana została metoda \lstinline|read_line()| czytająca linia po linii. Obsługuje ona zakończenia linii \lstinline|\n| oraz \lstinline|\r\n|. Wykorzystywana jest ona przez moduł list oraz, aby uniknąć duplikacji kodu, przez klienta HTTP do przetwarzania odpowiedzi.
		
		\subsection{Listy}
			Listy implementują wczytywanie różnych formatów. Przechowują także wskaźniki na bufory danych, które są wspólne dla każdej implementacji. Klasa \lstinline|List| jest klasą nadrzędną. Odczytuje ona źródło danych linia po linii, a~następnie przekazuje do funkcji \lstinline|List::consume_line_format()| której implementacja zawarta jest w~podklasach opisanych poniżej. Dokonuje ona właściwej interpretacji formatu.
			
			\paragraph{Statyczna alokacja} Wszystkie klasy list są tworzone statycznie i~współdzielą pamięć wpisów. Jest to bardzo istotnie w~środowisku o~ograniczonej pamięci operacyjnej. Zazwyczaj ładowanie jednego typu wyników nie koliduje z~pozostałymi, a~jeżeli tak się dzieje to resetowany jest stan programu ładującego który przy następnym użyciu musi przeładować dane listy, bo zostały one nadpisane. Z~tego też powodu modyfikowanie listy ulubionych odbywa się bezpośrednio w~pamięci stałej, bez ładowania całej listy stacji. Statyczna alokacja ma też swoje pozytywne strony, mianowicie przechowywany jest stan list. Po powrocie z~wybranej stacji do wyników ładowania (do listy ulubionych lub wyszukiwania) nie ma potrzeby od nowa ładować danych.
			
			\subsubsection{Format M3U}
				Jednym z~formatów obsługiwanych przez oprogramowanie odbiornika jest format M3U\textsuperscript{\cite{m3u}} w~wersji \textit{Extended}. Format ten został wybrany z~uwagi na jego prostotę budowy oraz to, że strony agregujące stacje radiowe go obsługują. Jest to popularny format zapisu list odtwarzania. W~wersji standardowej przybiera postać linków do plików/strumieni zapisanych w~osobnych liniach pliku tekstowego. Wersja rozszerzona wprowadza specjalne komentarze. Zaimplementowany interpreter obsługuje komentarz \lstinline|EXTINF|, który zawiera czas trwania ścieżki oraz jej nazwę.
			
			\subsubsection{Format PLS}
				Niektóre stacje radiowe przed wysłaniem strumienia audio, wysyłają plik w~formacie PLS\textsuperscript{\cite{pls}}. Plik taki zawiera adresy kilku serwerów strumieniujących tą samą stację.
				Klient powinien połączyć się z losowo wybranym serwerem w~celu uniknięcia nadmiernego obciążenia jednego z~nich.
				Po wykryciu tego formatu ładowane są pierwsze 2~stacje (konfigurowalne) i~wybierana jest jedna z~nich. Format składa się z~par deklaracji postaci \lstinline|FileX| i~\lstinline|TitleX|, gdzie \lstinline|X| to kolejne liczby naturalne.
			
		% TODO \subsection{Wpis na liście}
		% TODO \subsection{Pliki pomocnicze}
			
		\subsection{Klasy ładujące}
			Wczytywanie list z~konkretnych źródeł danych implementowane jest przez podklasy klasy \lstinline|Loader|. Prezentuje ona spójny interfejs do ładowania danych. Metody \lstinline|load()| oraz \lstinline|load_abort()| umożliwiają zarządzanie tym procesem. Tworzone jest zadanie systemowe, które wykonuje potrzebne kroki bez blokowania interfejsu użytkownika. Funkcje \lstinline|get_entry_count()| oraz \lstinline|get_entry()| pozwalają na pobieranie informacji o~wynikach. Klasa obsługuje także paginację wyników. Podklasy mogą definiować własne funkcje \lstinline|begin()|, które pełnią rolę konstruktora obiektu (przekazują niezbędne parametry, ale bez dynamicznej alokacji pamięci). Mogą one wymagać podania ścieżki lub np. słowa kluczowego do wyszukania.
			
			\subsubsection{Wyniki wyszukiwania}
				Pierwszą implementacją klasy ładującej była klasa służąca do wczytania wyników wyszukiwania stacji \lstinline|LoaderSearch|. Poprzez metodę \lstinline|begin(query)| przekazywana jest jej fraza do wyszukania. Następnie następuje odwołanie do klienta HTTP, który implementuje interfejs \lstinline|DataSource|. Agregat stacji radiowych zwraca dane wraz z~nagłówkiem \lstinline|Content-Type|. Jego wartość determinuje, która lista zostanie użyta do interpretacji. Typ \lstinline|audio/mpegurl| oznacza \lstinline|ListM3U|, natomiast \lstinline|audio/scpls| lub \lstinline|audio/x-scpls| implikuje wykorzystanie \lstinline|ListPLS|. Tak załadowana lista stacji jest dostępna dla reszty systemu poprzez metody klasy nadrzędnej \lstinline|Loader|.
			
			\subsubsection{Ulubione stacje}
				Z~uwagi na gotową implementację formatu M3U, zdecydowano się na wykorzystane go do przechowywania listy ulubionych stacji. Prostota formy oznaczała że możliwe będzie dodawanie wpisów w~miejscu bez konieczności ładowania całego pliku do pamięci. Kolejną zaletą jest fakt, że pliki \lstinline|.m3u| to pliki tekstowe. Przewidziano możliwość zaimplementowania funkcji eksportowania i~importowania listy ulubionych na kartę SD. Użytkownik mógłby modyfikować plik we własnym zakresie dodając stacje niebędące w~ogólnodostępnym spisie stacji. Na ten moment nie zostało to zaimplementowane.

			\subsubsection{Znane sieci Wi-Fi}
				Tutaj także zostały wykorzystane pliki w~formacie M3U. Nazwa sieci przechowywana jest jako nazwa stacji, a~pole \lstinline|url| wykorzystano na przechowywanie hasła. Lista ta, razem z~listą ulubionych stacji używa klasy bazowej \lstinline|LoaderM3U|. Dzięki temu implementacja interpretacji formatu jest jedna. Wczytywanie listy następuje w~ekranie ustawień, jak i~po uruchomieniu urządzenia, w~celu automatycznego połączenia z~najlepszą siecią.
				
			\subsubsection{Wykryte sieci Wi-Fi}
				Do skanowania pobliskich sieci została stworzona implementacja \lstinline|LoaderWifiScan|. Uruchamia ona skanowanie sieci przy pomocy wbudowanego w~środowisko \lstinline|pico-sdk| sterownika układu Wi-Fi. Zwracane sieci są w~losowej kolejności, często zduplikowane. Aby temu zaradzić, jak i~obsłużyć stronicowane wyników, stworzona została klasa \lstinline|LFSorter|, opisana poniżej.
				
			\subsubsection{Pliki lokalne}
				Interfejs list i~klas ładujących okazał się także przydatny przy ładowaniu listy lokalnych plików z~karty SD. Klasa \lstinline|LoaderLocal| wykorzystuje referencję do klasy \lstinline|Path|, która implementuje również metody poruszania się po strukturze folderów. Metoda \lstinline|go()| powoduje zejście w~dół struktury drzewiastej, natomiast \lstinline|up()| umożliwia przejście poziom wyżej. Metody obsługi systemu plików zwracają listingi katalogów w~kolejności dodawania plików do niego. Porządek alfabetyczny, jak i~priorytet folderów został osiągnięty przy wykorzystaniu klasy \lstinline|LFSorter|.
				
			\subsubsection{Stałe wyniki}
				Bardzo prosta klasa ładująca, przepisuje ona jedynie wyniki z~pamięci stałej do listy wyników. Używana przez ekran ustawień.
				
%		\paragraph{Karta SD -- FatFS}
		\subsection{Pamięć Flash -- LittleFS}
			Aby lista ulubionych stacji czy lista zapisanych sieci Wi-Fi były przechowywane pomiędzy wyłączeniami urządzenia, musiały one być zapisane w~pamięci nieulotnej. Mikrokontroler RP2040 przechowuje program w~zewnętrznej pamięci typu \textit{flash}. Na płytkach Pi Pico~W zamontowane zostały układy pamięci o~rozmiarze 2MB. Środowisko programistyczne umożliwia, oprócz oczywistej operacji odczytu, także zapis do tej pamięci. Wykorzystano tą właściwość do umieszczenia w~końcowych 64KB systemu plików LittleFS\textsuperscript{\cite{littlefs}}, specjalnie stworzonego do przechowywania danych w~środowiskach wbudowanych i~pamięciach typu \textit{flash}. Stworzona została klasa \lstinline|LFSAccess|, która dostarcza podstawowe metody do manipulacji tekstami, jak również implementuje interfejs \lstinline|DataSource|.
			
		\subsection{Sortowanie -- LFSorter}
			Niektóre metody zwracają wyniki w~postaci strumienia kolejnych wyników, bez możliwości indeksacji indywidualnych pozycji. Z~powodu ograniczonej pamięci RAM, nie jest możliwe wczytanie wszystkich wyników i~wybranie z~nich kilku w~zadanej kolejności. Ten problem rozwiązuje klasa \lstinline|LFSorter|, która zapisuje cały strumień do pamięci nieulotnej \textit{flash} (przy użyciu \lstinline|LFSAccess|). Następnie, przy użyciu metody \lstinline|get_smallest_n_skip_k(n, k)|, pozwala wczytać elementy $k,\; k+1,\; ...,\; k+n-1$ z~ciągu posortowanych elementów (indeksowanie od 0, sortowanie wg dostarczonej funkcji porównującej). Ciąg ten nie jest nigdzie zapisywany.
			
			\paragraph{Pamięć podręczna}
			Plik tymczasowy utworzony podczas procesu ładowania strumienia, może zostać wykorzystany w~kolejnych wywołaniach. Np. z~każdą nową stroną nie ma potrzeby skanować od nowa dostępnych sieci Wi-Fi czy ładować listingu lokalnego katalogu.
						
			\paragraph{Złożoność obliczeniowa}
			Wewnętrznie implementowany jest algorytm sortowania przez wybieranie, który ignoruje pierwsze $k$ elementów, a~następnie zwraca kolejne $n$ elementów. Np. przy ładowaniu pierwszej strony wyników $k$ będzie równe 0. Liczba $n$ jest stała i~równa maksymalnej liczbie elementów na jednej stronie. Jeżeli założymy, że liczba elementów w~całym strumieniu będzie rzędu $k+n$ (np. ładujemy ostatnią stronę) to złożoność obliczeniowa algorytmu to $O((k+n)^2)$. Nie jest to efektywny algorytm, jednak pozwala maksymalnie zaoszczędzić pamięć RAM, gdyż jej użycie nie zależy od liczby wszystkich elementów.
			

		\subsection{Współpraca klas ze sobą}
			Na rysunku~\ref{3/PicoRadio-loaders} przedstawiono współpracę poszczególnych klas. Klasy ładujące mogą wykorzystywać listy. Jeżeli to robią, ich źródło danych musi implementować interfejs \lstinline|DataSource|. Tak się dzieje w~przypadku klas \lstinline|LoaderM3U| oraz \lstinline|LoaderSearch|. Ładują one dane odpowiednio z~pamięci stałej oraz z~Internetu. Interpretacja zachodzi przy użyciu odpowiednich list. Odmienna sytuacja ma miejsce w~przypadku pozostałych klas. Nie interpretują one plików/strumieni, więc nie potrzebują być implementacjami \lstinline|DataSource|, ani używać list. Interfejsy do skanowania Wi-Fi i~ładowania plików lokalnych zwracają nazwy kolejnych elementów bezpośrednio, bez konieczności interpretacji. Jednak są one losowe i~wymagają posortowania przy wykorzystaniu \lstinline|LFSSorter|. Klasa \lstinline|LoaderConst| bezpośrednio przepisuje wyniki i~jest najprostszą ze wszystkich. Służy do definiowania list o~stałej zawartości.
			
			\imgh{3/PicoRadio-loaders}{Współpraca list z~klasami ładującymi i~ekranami}{0.95}
			
			\paragraph{Ekrany}
				Istnieje powiązanie pomiędzy klasami ładującymi a~ekranami. Każdej klasie ładującej przyporządkowany jest ekran, który wyświetla dane pozycje. Zawiera on w~sobie referencję klasy ładującej. Przekazuje ją swojej klasie nadrzędnej jako wskaźnik do klasy ogólnej \lstinline|Loader|. Tym sposobem, każdy z~ekranów może ustalać szczegółowe parametry ładowania, a~ekran nadrzędny implementuje ogólne funkcje związane z~wyświetlaniem załadowanych pozycji na ekranie.
	
	\section{Obsługa ekranu LCD}
		\label{sec:screen}
		Z~uwagi na założenie miniaturyzacji urządzenia wybrany został relatywnie mały ekran. Urządzenie firmy Adafruit zawiera wyświetlacz ciekłokrystaliczny (\textit{LCD}) o~przekątnej 1.8 cala oraz sterownik ST7735S obsługiwany poprzez protokół SPI. Na ekranie znajdowała się folia z~zieloną zakładką. Jest to istotne ponieważ na rynku występują także wersje oznaczone kolorem czerwonym, posiadające inne opcje programowania.
		$ $\\
		
		\noindent
		Moduł wyświetlacza został podzielony na następujące podmoduły:
		\begin{itemize}
			\item \lstinline|tft| -- niskopoziomowa obsługa komunikacji i~rysowania,
			\item \lstinline|assets| -- zasoby do wyświetlania ikon i~czcionek,
			\item \lstinline|screens| -- ekrany użytkownika
			\begin{itemize}
				\item \lstinline|wifi| -- ekrany Wi-Fi
			\end{itemize}
			\item \lstinline|screenvirt| -- ekrany wirtualne.
		\end{itemize}
		
		\subsection{Komunikacja niskopoziomowa}
			Kod niskopoziomowej komunikacji ze sterownikiem ekranu zawiera folder \lstinline|display/tft|. Inicjalizacja zapożyczona została z~biblioteki firmy Adafruit. W~pliku \lstinline|st7735_init.h| znajdują się definicje komend oraz listy poleceń potrzebne do uruchomienia wyświetlacza. Klasa \lstinline|ST7735S| zawiera metody \lstinline|reset()|, \lstinline|write_command_list()| oraz \lstinline|module_init()|, które odpowiadają za ustawienie poprawnych parametrów pracy urządzenia. Wszystkie wymagane operacje, włącznie z~konfiguracją pinów mikrokontrolera, zawiera metoda \lstinline|ST7735S::init()|.
			
		\paragraph{Kolory}
			Do każdej funkcji przekazywany jest kolor w~formacie 24~bitowego RGB, tj. po 8~bitów na kolor, w~postaci jednej liczby. Kolor ten jest automatycznie mapowany na wewnętrzną 16~bitową reprezentację obsługiwaną przez sterownik. Podział na składowe RGB w~tej postaci to:
			\begin{itemize}
				\item 5~bitów -- kolor czerwony (najstarsze),
				\item 6~bitów -- kolor zielony,
				\item 5~bitów -- kolor niebieski.
			\end{itemize}
		
		\paragraph{Współrzędne}
			Na rysunku~\ref{3/PicoRadio-screen-xy} przedstawiono sposób, w~jaki na ekranie reprezentowane są współrzędne. Na magistralę szeregową dane wysyłane są linia po linii.
			
		\imgh{3/PicoRadio-screen-xy}{Układ współrzędnych na ekranie}{0.45}
		
		\subsection{Rysowanie}
			Aby przesłać kolory dla wybranych pikseli na ekranie, należy ustawić odpowiednie zmienne wskaźnikowe wewnątrz sterownika wyświetlacza. Funkcja \lstinline|ST7735S::setup_write()| pozwala na ich ustawienie i~przekazanie początkowych oraz końcowych wartości współrzędnych. Dzięki temu na ekranie tworzony jest prostokąt dla którego należy przesłać zaktualizowane wartości pikseli w~formacie podanym powyżej. Jednak oprócz funkcji do obsługi niskopoziomowej komunikacji klasa \lstinline|ST7735S|  wspiera także rysowanie prostokątów, czcionek i~ikon.
			
			\subsubsection{Prostokąty}
				Na rysunku~\ref{3/PicoRadio-screen-rect} przedstawiono jak rysowane są prostokąty. Używana jest funkcja \lstinline|fill_rect(x, y, w, h, bg)|, która przyjmuje następujące argumenty:
				\begin{itemize}
					\item \lstinline|x| -- współrzędna kolumny ,
					\item \lstinline|y| -- współrzędna wiersza,
					\item \lstinline|w| -- szerokość,
					\item \lstinline|h| -- wysokość,
					\item \lstinline|bg| -- kolor tła (24 bitowe RGB).
				\end{itemize}
				
				\imgh{3/PicoRadio-screen-rect}{Współrzędne prostokąta na ekranie}{0.6}
			
			\subsubsection{Czcionki}
				Każda czcionka składa się z dwóch tablic \lstinline|ascii_data_H| i~\lstinline|utf8_data_H|, gdzie \lstinline|H| to rozmiar czcionki, tj. wysokość w~pikselach (szerokość jest połową wysokości). Zostały one opisane strukturą \lstinline|font| w~pliku \lstinline|assets/font.hpp|.
				
				\paragraph{Mapowanie}
					Pierwsza tablica zawiera znaki od spacji (kod \lstinline|0x20| ASCII) do tyldy (kod \lstinline|0x7E|). Druga z~tablic zapewnia wyświetlanie znaków Unicode. Mapowanie jest tu arbitralne zapewnione funkcją sprawdzającą wartości 2-3 kolejnych bajtów danych. Zawiera ona obecnie 18~polskich znaków (wariant duży i~mały) jak również symbol nieskończoności. Istnieje możliwość dodawania dowolnej liczby nowych znaków.
				
				\paragraph{Kodowanie}
					Czcionki kodowane są jako bitmapy. Każda z~tablic zawiera tyle elementów, ile jest pikseli we wszystkich znakach czcionki w~tej tablicy. Każda wartość to jasność piksela (255 -- kolor pierwszoplanowy, 0~-- kolor tła). Dla przykładu czcionka podstawowa ASCII zawierająca 95~znaków o~wymiarach 24x12 pikseli będzie miała rozmiar 27360~bajtów. Dodając 19~znaków Unicode otrzymujemy 32832~bajtów. Kodowanie znaku \lstinline|A| z~tej czcionki przedstawia rysunek~\ref{3/PicoRadio-screen-font}.
				
				\imgh{3/PicoRadio-screen-font}{Kodowanie czcionek (wartości 0x00 zastąpiono pustymi znakami)}{0.6}
				
				W~projekcie została wykorzystana czcionka Ubuntu\textsuperscript{\cite{ubuntu_font}} w~wersji Mono (stała szerokość każdego znaku). Stworzony skrypt w~języku Python wygenerował potrzebne bitmapy do projektu. Dostępne rozmiary czcionki to 12~(rozmiar ok. 8kB), 16~(ok. 14kB) i~24~(ok. 32kB). Razem stanowi to 54kB, czyli 2.65\% pamięci ROM. Czcionka została zapisana w~pliku \lstinline|assets/ubuntu_mono.cpp| a~także utworzono funkcję dostępową \lstinline|ubuntu_font_get_size()|. 
				
			\subsubsection{Ikony}
				Ikony kodowane są w~podobny sposób do czcionek. Różnią się tym, że każda ikona może mieć inny rozmiar i~jest kodowana swoją własną tablicą. W~projekcie znalazły się m.in. ikony:
				\begin{itemize}
					\item \lstinline|icon_backspace| -- 15x15, ikona usuwania tekstu,
					\item \lstinline|icon_back| -- 11x11, ikona wstecz,
					\item \lstinline|icon_fav_back| -- 15x16, ikona powrotu do listy ulubionych.
				\end{itemize}
		
		\subsection{Ekrany}
			Na rysunku \ref{3/PicoRadio-screens} przedstawiono hierarchię ekranów. Klasą nadrzędną wszystkich ekranów jest klasa \lstinline|Screen|. Implementuje ona podstawową siatkę przycisków, przechodzenie pomiędzy nimi, jak również rysowanie i~aktualizację ruchomych tekstów. Bezpośrednimi potomkami tej klasy są ekrany o~niestandardowym układzie (np. ekran odtwarzania). Jednak większość ekranów jest podklasami dwóch innych potomnych klas: \lstinline|ScreenList| i~\lstinline|ScreenKb|. Są to klasy wirtualne (znajdują się w~katalogu \lstinline|screenvirt|), niemogące istnieć bezpośrednio. Standaryzują one układ potomnych ekranów. Pierwsza z~nich implementuje widok listy, a~druga, widok klawiatury. \lstinline|ScreenKb| implementowana jest przez ekran wyszukiwania stacji i~ekran wprowadzania hasła Wi-Fi.
			
			\imgh{3/PicoRadio-screens}{Hierarchia ekranów}{0.55} % TODO opisać te ekrany na rysunku np ScPlay -- odtwarzacz
			
		\subsection{Menedżer ekranów}
			Ekranami użytkownika zarządza plik \lstinline|screenmng.cpp|. Tworzy on szereg konstrukcji wymaganych do poprawnego obsłużenia interfejsu użytkownika. Poniżej zostały opisane wybrane z~nich.
			
			\paragraph{Semafory}
				Aby zabezpieczyć dostęp do medium szeregowego, utworzony został semafor binarny \lstinline|mutex_display|. Zapewnia on ciągłość operacji przesyłu danych. Musi ona wystąpić ponieważ w~sterowniku został wybrany prostokąt, którego kolory będą aktualizowane. Dodatkowo powstał \lstinline|mutex_ticker|, który zabezpiecza funkcję aktualizacji ekranów. Zadanie \lstinline|scr tick| jest wywoływane co 50ms i~aktualizuje np. położenie ruchomych tekstów na ekranie.
				
			\paragraph{Instancje klas}
				W~pliku zdefiniowano także instancje najważniejszych klas. Zdefiniowana jest instancja klasy \lstinline|ST7735S| oraz klas ładujących \lstinline|LoaderSearch|, \lstinline|LoaderFav| i~\lstinline|LoaderLocal|. Statyczna definicja pozwala na analizę zużycia pamięci w~trakcie kompilacji programu oraz chroni przed niedeterministyczną operacją rezerwowania pamięci w~czasie działania programu.
				
			\paragraph{Obsługa ekranów użytkownika}
				Poszczególne ekrany użytkowników także zostały tutaj zainstancjonowane. Plik nagłówkowy \lstinline|screenmng.hpp| eksportuje te definicje, aby ekrany mogły ustawiać parametry innych ekranów, które otwierają. Znalazły się tutaj także funkcje do zmiany aktualnie wyświetlanego ekranu. Domyślnie zmiana ta odbywa się poprzez funkcję \lstinline|screenmng_input()|, do której przekazywane są naciśnięcia przycisków. Klasa bazowa wszystkich ekranów zawiera metodę \lstinline|Screen::input()|, która może zwrócić wskaźnik do ekranu, który ma zostać otworzony jako następny. Jest ona wywoływana przez funkcję zarządcy i~jeżeli zwróci niezerowy wskaźnik to otwierany jest nowy ekran. Jednak w~trakcie rozwoju oprogramowania, konieczne stało się zdefiniowanie funkcji \lstinline|screenmng_open()|, która może arbitralnie zmieniać otwarty ekran. Jest to używane np. przy zakończeniu odtwarzania. Wskaźnik \lstinline|current_screen| ustawiany jest zawsze na aktualnie otwarty ekran. Jest to użyteczne przy funkcji aktualizacji ekranu.

	\section{Odczyt informacji wejściowych}
		\noindent
		Zawartość modułu znajduje się w 3~katalogach:
		\begin{itemize}
			\item \lstinline|buttons| -- Przyciski, interakcja z~użytkownikiem,
			\item \lstinline|sd| -- Wykrywanie karty SD,
			\item \lstinline|analog| -- Odczyt poziomu naładowania baterii.
		\end{itemize}
		
		Katalogi te zawierają po 2~pliki (nagłówkowy i~kod źródłowy) realizujące funkcje opisane poniżej.
		
		\subsection{Interakcje użytkownika}
			Dane od użytkownika pobierane są za pomocą przycisków. Każdy z~nich dołączony jest bezpośrednio do mikrokontrolera. Naciśnięcie przycisku powoduje wygenerowanie zbocza opadającego. Wykryte przez układ, aktywuje ono przerwanie. Eliminacja drgań styków została osiągnięta poprzez zastosowanie zewnętrznych kondensatorów i~wbudowanych przerzutników Schmitta dostępnych na każdym wejściu RP2040, oraz poprzez odczekanie 20ms po wywołaniu przerwania i~ponownym sprawdzeniu stanu na wejściu. Przerwanie, zgodnie z~dobrymi praktykami, ustawia jedynie odpowiednie flagi, które zostały opisane poniżej.
			
			\paragraph{Przekazywanie danych}
				Podsystem przycisków opiera się o dwie tablice zawierające odpowiednie dane dla każdego przycisku z~osobna. Pierwsza z~nich, o~nazwie \lstinline|b_pressed[]|, zawiera wartości logiczne. Mówią one o~tym czy przycisk jest w~danej chwili wciśnięty czy nie. Druga, \lstinline|b_pressed_time_us[]|, zawiera informację, w~której mikrosekundzie nastąpiło wciśnięcie przycisku. Używa ona wbudowanej funkcji \lstinline|time_us_32()| do pobierania czasu.
			$ $\\
			
			Przerwanie, oprócz ustawienia odpowiednich wartości w~tablicach, wybudza zadanie \lstinline|input handle|. Zawiera ono maszynę stanów, która umożliwia, oprócz podstawowego przekazywania zdarzeń do interfejsu, powtarzanie naciśnięć, tak jak robią to klawiatury komputerowe. Dodatkowo zawarto w~niej także obsługę wygaszania ekranu. Na rysunku~\ref{3/PicoRadio-buttons} przedstawiono schemat blokowy ilustrujący jej funkcjonowanie.
			
			\imgh{3/PicoRadio-buttons}{Maszyna stanów w~module przycisków}{0.75}
			
			\paragraph{Czas oczekiwania}
				Czas oczekiwania na ponowne uruchomienie maszyny jest tym mniejszy im częściej dane zdarzenie może wystąpić. To znaczy, gdy żaden z~przycisków nie jest naciśnięty, a~ekran wygaszony, maszyna oczekuje w~nieskończoność na sygnał z~przerwania. Gdy ekran nie jest wygaszony, zadanie czeka co najwyżej 30s. Po tym czasie następuje sprawdzenie warunków do wygaszenia, np. czy odtwarzanie jest uruchomione (pominięte na rysunku). Gdy są one spełnione, ekran zostaje wygaszony. Kiedy użytkownik kliknął i~przytrzymuje przycisk, zadanie wykonuje się 26~razy na sekundę. Zapewnia to wystarczająco dokładny pomiar czasu, a~nie obciąża zbytnio procesora. Jeżeli stan wejścia jest niski po minimum 20ms wykonywana jest jednokrotna akcja. Rozpoczęcie powtarzania następuje po 500ms ciągłego naciskania. Wartość 26~definiuje prędkość powtarzania znaków. Jest to częstotliwość powtarzania znaków przez klawiatury komputerowe. Wszystkie czasy są konfigurowalne w~momencie kompilacji programu.
		
		\subsection{Karta SD}
			Pin \textit{Card Detect} slotu na karty SD również został bezpośrednio podłączony do mikrokontrolera i~działa zgodnie z~tymi samymi zasadami co piny przycisków. Jednak obsługa wykrywania karty jest znacznie prostsza. Zadanie \lstinline|sd| wykrywa zmianę stanu na odpowiednim pinie i~montuje lub odmontowuje kartę SD z~wykorzystaniem biblioteki FatFS.
			
		\subsection{Bateria}
			Poziom naładowania baterii jest określany przez pomiar napięcia na jednym z~wejść wewnętrznego przetwornika analogowo-cyfrowego mikrokontrolera RP2040.
			Na wejściu zastosowano dzielnik napięcia, współczynnik podziału wynosi 2.
			Dzięki niemu przy maksymalnym napięciu ogniwa wynoszącym 4,2V na wejście przetwornika podawane jest 2,1V co mieści się w~zakresie poprawnych napięć wejściowych przetwornika (0 - 3,3V).
			W~pliku \lstinline|analog.cpp| zaimplementowano funkcję \lstinline|battery_voltage()|, która zwraca napięcie ogniwa po wszystkich niezbędnych przekształceniach. Dodatkowo, kod zawiera funkcję zwracającą aproksymowany poziom naładowania baterii w~procentach. Funkcja \lstinline|battery_percentage()| zwraca liczbę z zakresu 0~- 100, która oznacza poziom naładowania ogniwa. Na podstawie wyniku tej funkcji rysowana jest ikona baterii w~górnym pasku ikon statusowych. Ma ona kolor zielony i~jest w~pełni naładowana gdy bateria posiada 60\% lub więcej energii, 20\% i~więcej to kolor żółty, natomiast poniżej 20\% ma ona kolor czerwony.
		
	\section{Bezprzewodowy dostęp do Internetu}
		Obsługa sieci bezprzewodowej podzielona została na 3~pod-moduły:
		\begin{itemize}
			\item \lstinline|wificonnect| -- połączenie z~Wi-Fi,
			\item \lstinline|wifiscan| -- skanowanie dostępnych sieci,
			\item \lstinline|wifibest| -- znalezienie najlepszej sieci do połączenia po uruchomieniu.
		\end{itemize}
		
		Każdy z~tych modułów dodaje nowe funkcje do przestrzeni nazw \lstinline|wifi|.
		
		\subsection{Połączenie z~siecią Wi-Fi}
			Głównym zadaniem tej części modułu sieciowego jest pierwsze połączenie z~siecią Wi-Fi po jej manualnym wyszukaniu i~podaniu hasła. Służy do tego funkcja \lstinline|connect_async()|. Tworzy ona zadanie systemu operacyjnego \lstinline|wifi conn|, które próbuje połączyć się z~daną siecią. Postęp procesu jest aktualizowany na ekranie przy pomocy wskaźników do funkcji przekazanych w~strukturze \lstinline|cb_fns|. Są to:
			\begin{itemize}
				\item \lstinline|upd| -- ogólna aktualizacja, zawiera opis wykonywanego aktualnie kroku,
				\item \lstinline|scan| -- aktualizacja po znalezieniu sieci o~podanej nazwie, zawiera siłę sygnału,
				\item \lstinline|conn| -- wywoływane po pomyślnym podłączeniu do sieci.
			\end{itemize}
			
			Odwołanie \lstinline|conn| powoduje także zapisanie, do przyszłego użycia, identyfikatora sieci wraz z~hasłem dostępu.
			Dodatkowe funkcje w~tym module umożliwiają inicjalizację podsystemu Wi-Fi oraz np. sprawdzenie stanu połączenia.
			
		\subsection{Skanowanie dostępnych sieci}
%			Wykorzystywany przez ekran \lstinline|ScWifiScan| oraz klasę ładującą \lstinline|LoaderWifiScan|,
			Moduł ten umożliwia wyszukanie metodą \lstinline|scan()| pobliskich sieci. Wyniki skanowania są zapisywane do pamięci nieulotnej przy użyciu przekazanej instancji \lstinline|LFSAccess|. Następnie ładowanie kolejnych stron wyników umożliwia funkcja \lstinline|read(n, k)|, która działa analogicznie do funkcji \lstinline|LFSorter::get_smallest_n_skip_k(n, k)|. Sortowanie odbywa się nierosnąco wg siły sygnału.
			
		\subsection{Połączenie po uruchomieniu}
			Gdy urządzenie zostanie uruchomione, użytkownik oczekuje że samo odnajdzie ono najlepszą do połączenia sieć. Zachowanie takie implementuje funkcja \lstinline|connect_best_saved()|. Wykorzystuje ona funkcje \lstinline|scan()| i~\lstinline|read()| żeby pojedynczo wczytywać sieci, poczynając od tej o~najwyższej sile sygnału. Gdy sieć istnieje w~zapisanych sieciach, inicjowane jest połączenie. W~przeciwnym wypadku ładowana jest następna sieć i~tak aż do skutku, albo do wyczerpania się listy wykrytych sieci.
			
\cleardoublepage
\chapter{Organizacja pracy}
	W~tym rozdziale opisano organizację i~przebieg prac nad projektem.
	
	\section{Przebieg prac}
		Prace nad projektem rozpoczęły się w~kwietniu 2023~roku.
		Projekt realizowany był etapami.
		Diagram ilustrujący ich umiejscowienie w~czasie pokazano na rysunku~\ref{4/PicoRadio-steps}.
		Etapy te zostały opisane poniżej.
		
		\imgh{4/PicoRadio-steps}{Etapy projektu}{1}
		
		\subsection{Określenie wymagań sprzętowych}
			Propozycja wykorzystania układu RP2040 była względnie ryzykowna. Rdzenie Cortex M0+ mogły okazać się niewystarczające do obsłużenia wszystkich zadań, zwłaszcza, że brak im jednostki obliczeń zmiennoprzecinkowych. Należało więc jak najszybciej określić czy poradzą sobie one z~najważniejszym i~najbardziej wymagającym aspektem jakim jest dekodowanie formatu MP3.
			$ $\\
			
			W~krótkim czasie powstał wstępny prototyp przedstawiony na rysunku~\ref{4/prototype_1}. Zmontowany na płytce stykowej, zawierał minimum elementów niezbędnych do testowania odtwarzania plików w~formacie MP3. Obok modułu Raspberry Pi Pico~W znalazły się tam układy przetwornika cyfrowo-analogowego i~wzmacniacza, oraz slot na kartę SD.
			
			\imgh{4/prototype_1}{Pierwszy prototyp do testów odtwarzania dźwięku}{0.6}
			
			Początkowo projekt odtwarzał tylko nieskompresowane pliki WAV zapisane na karcie SD. Na listingu~\ref{lst/4/prototype.c} przedstawiono jak odbywało się doładowywanie bufora danymi. Flagi \lstinline|a_done_irq| i~\lstinline|b_done_irq| były ustawianie w~przerwaniach DMA. Główna pętla programu oczekiwała na te przerwania i~ładowała dodatkowe dane.
			
			% commit: https://github.com/MrJake222/pico-radio/commit/f74a7883a8a824952876943f6f33ca1d82e739dd
			\lstfile{c}{Realizacja podstawowego ładowania bufora}{lst/4/prototype.c}
			
			Dostęp do karty SD, jak i~komunikacja \isqs{} z~przetwornikiem działały poprawnie. Dzięki makrom \lstinline|DBG_ON| i~\lstinline|DBG_OFF| ustawiającym pin mikrokontrolera w~odpowiedni sposób, możliwa była zewnętrzna kontrola czasu wykonania odczytu i~(później) dekodowania.
			\pagebreak
			
			Następnie, przy wykorzystaniu biblioteki \lstinline|minimp3|\textsuperscript{\cite{minimp3}}, dodana została obsługa dekodowania MP3. Krótki fragment kodu na listingu~\ref{lst/4/benchmark.c} przedstawia w~jaki sposób badano czas dekodowania pojedynczej ramki audio. Niestety czas przetworzenia jednej ramki wynosił ok. 90ms.
			Czas ten był zdecydowanie zbyt długi gdyż uzyskane w~ten sposób dane audio  były odtwarzane w~ciągu około 26ms.

			% commit: https://github.com/MrJake222/pico-radio/commit/798cda3f2b81ec04788e7fbe123bc1dbff969250
			\lstfile{c}{Pomiar czasu dekodowania ramki MP3}{lst/4/benchmark.c}
			
			% commit: https://github.com/MrJake222/pico-radio/commit/ede28044f02faab5ef6a6204bf7e402d5687e736
			Koniecznym stało się znalezienie innej biblioteki dekodującej strumień MP3, która pozwoliłaby skrócić czas dekodowania pojedynczej ramki do poziomu poniżej 26ms. W~trakcie poszukiwań zwrócono uwagę na bibliotekę \lstinline|helixmp3|, która w~swojej implementacji wykorzystywała arytmetykę stałopozycyjną. Zastosowanie tej biblioteki skróciło czas dekodowania pojedynczej ramki MP3 do około 20ms, co oznaczało możliwość kontynuacji projektu według pierwotnej koncepcji wykorzystania programowego dekodera strumienia MP3.
			
		\subsection{Standaryzacja interfejsu i~radio internetowe}
			Pod koniec kwietnia w~repozytorium projektu znajdowało się wiele dosyć luźno powiązanych ze sobą fragmentów kodu, które były wykorzystane do sprawdzenia pojedynczych funkcjonalności.
			Przystąpiono do prac nad refaktoryzacją i~integracją opracowanych wcześniej rozwiązań, a~także dodaniem nowych by finalnie uzyskać w~maju oprogramowanie układowe dla Raspberry Pi Pico~W realizujące funkcje odtwarzania strumienia stacji radiowej.
			$ $\\
			
%			W~trakcie standaryzacji interfejsu odtwarzanie plików MP3 oraz WAV zostało zintegrowane i~obsługiwane jest obecnie przez jedną wspólną funkcję.
%			Zaimplementowany i~zintegrowany został także klient HTTP, a~wraz z~nim obsługa strumieni internetowych.
%			Pod koniec maja odtwarzacz plików MP3, za sprawą powyższych zmian, stał się radiem internetowym.
			
		\subsection{System operacyjny}
			W~czerwcu rozpoczęto pracę związane z~interfejsem sterującym pracą odbiornika.
			W~drugiej wersji prototypu widocznej na rysunku~\ref{4/prototype_2} do projektu został dołączony wyświetlacz oraz przyciski. Zaistniała potrzeba obsługi tych peryferiów, przy jednoczesnym odbiorze i~przetwarzaniu danych związanych z~odtwarzaniem stacji radiowej.
			$ $\\
			
			Dosyć szybko okazało się, że samodzielne stworzenie oprogramowania, które będzie równocześnie wykonywało wiele funkcji, efektywnie wykorzystując przy tym zasoby Raspberry Pi Pico~W może się okazać dużym wyzwaniem mogącym negatywnie wpłynąć na czas realizacji projektu.
			
			\imgh{4/prototype_2}{Prototyp z~wyświetlaczem i~klawiaturą}{0.8}
			
			Zdecydowano się wykorzystać do tego celu dedykowany dla mikrokontrolerów system operacyjny czasu rzeczywistego FreeRTOS. Wymagało to wprowadzenia zmian w~projekcie, jednak uzyskane w~ten sposób korzyści były znaczące.
			Najważniejszą z nich była możliwość utworzenia zadań systemu operacyjnego, które współdzieląc czas procesora mogą wykonywać oddzielne czynności.
			Należą do nich m.in. obsługa wyświetlacza i~przycisków czy odbiór danych i~odtwarzanie dźwięku.
			Niestety pojawiły się również nowe problemy.
			
			\subsubsection{Problemy z~pamięcią RAM}
				Po dołączeniu systemu FreeRTOS z~domyślną konfiguracją okazało się, że w~czasie uruchomienia, brakuje pamięci RAM.
				Postanowiono wtedy używać jedynie statycznej alokacji pamięci, żeby zauważyć i~naprawić tego typu błędy już w~czasie kompilacji.
				$ $\\

				Powstało także narzędzie do analizy zużycia pamięci na podstawie pliku \lstinline|.elf.map|. Zawiera on informacje w~której części przestrzeni adresowej znajdują się poszczególne użyte w~projekcie zmienne oraz ile bajtów zajmują. Skrypt \lstinline|read-elf-map.py| w~języku Python analizuje ten plik. Wyświetla przyjazne dla użytkownika podsumowanie największych obiektów oraz procentową zajętość pamięci. Na listingu~\ref{lst/4/memory}, na którym widoczny jest wynik takiej analizy, widać, że w~finalnej wersji oprogramowania układowego pamięć została	użyta w~80\%, a~najwięcej pamięci alokują:
				\begin{itemize}
					\item 70kB (\lstinline|static.cpp|) -- bufor kołowy, pamięć wpisów list, instancja klienta HTTP,
					\item 49kB stos sieciowy
					\item 45kB (\lstinline|player.cpp|) -- bufor nieskompresowanego dźwięku, instancje dekoderów.
					\item 29kB system operacyjny FreeRTOS.
				\end{itemize}
				
				\lstfile{default}{Zajętość pamięci RAM}{lst/4/memory}
			
		\subsection{Interfejs użytkownika}
			Koniec czerwca to implementacja architektury ekranów użytkownika. Na początku została dodana obsługa ekranów klawiatury (rys.~\ref{4/first_screen_search}), wyników wyszukiwania (rys.~\ref{4/first_screen_res}) i~odtwarzacza. Wraz z~końcem miesiąca radio było gotowe do znajdowania i~odtwarzania stacji.
			
			\imghss{4/first_screen_search}{Ekran wyszukiwania z~klawiaturą}{4/first_screen_res}{Ekran wyników wyszukiwania}
			
		\subsection{Problem z~oknem TCP}
			W~trakcie trwających równolegle z~rozwojem oprogramowania testów odtwarzania stacji internetowych ujawniły się trudne do zidentyfikowania problemy.
			W~nieregularnych odstępach czasu odtwarzanie przestawało być płynne. Wypełnienie bufora spadało do zera, dźwięk zapętlał się. Objawy sugerowały potencjalne problemy w~komunikacji sieciowej pomiędzy serwerem stacji radiowej a~odbiornikiem.
			$ $\\
			
			Korzystając z~oprogramowania Wireshark\textsuperscript{\cite{wireshark}}, zapisywano przebieg komunikacji urządzenia z~serwerem.
			Zebrane dane zostały skorelowane z~poziomem zapełnienia bufora. Szczegółowa analiza wykazała, że źródłem problemu jest rozmiar okna TCP ogłaszany przez odbiornik.
			$ $\\
			
			Na rysunku~\ref{4/tcp_fail} przedstawiono jak prezentuje się ogłaszane przez interfejs \mbox{Wi-Fi} odbiornika okno TCP.
			Jego rozmiar jest odgórnie ograniczony do wartości 12kB.
			\imgh{4/tcp_fail}{Wykres rozmiaru okna TCP ogłaszany przez odbiornik}{0.85}
			
			Od czasu ok. 1:24 rozpoczęły się wspomniane problemy. Okazało się, że gdy bufor zaczyna się wyczerpywać, nie może on zostać wystarczająco szybko uzupełniony, ponieważ ilość danych na łączu ograniczona jest przez rozmiar okna TCP i~opóźnienie sieci.
			Problem został rozwiązany poprzez zwiększenie maksymalnego rozmiaru okna TCP do 56kB.
			$ $\\
			
			Skuteczność rozwiązania zweryfikowano przeprowadzając próby odbioru strumienia ze stacji z~której docierał on z~największym opóźnieniem. Wybrano do tego stacje nadająca z~Nowej Zelandii (opóźnienie 322ms). Wykres ogłaszanego przez odbiornik rozmiaru okna TCP podczas tej transmisji przedstawia rysunek~\ref{4/tcp_good}.
			
			\imgh{4/tcp_good}{Rozmiar okna TCP ogłaszany w~trakcie połączenia o znacznym opóźnieniu}{0.6}

			Jak widać, podobny problem zacząłby występować ok. 40~sekundy. Jednak rozmiar okna został zwiększony do ponad 40kB i~serwer szybko dostarczył brakujące dane.
%			Na uznanie zasługuje sugestia z~listy mailingowej projektu LwIP.
			%TODO https://lists.nongnu.org/archive/html/lwip-users/2023-07/msg00001.html
			
		\subsection{Proces przyrostowy}
			Po pozbyciu się uciążliwego problemu z~niewystarczającym rozmiarem okna TCP, skupiono się na dodaniu nowych funkcjonalności. W~lipcu powstały m.in:
			\begin{itemize}
				\item Wyświetlanie statystyk odtwarzania,
				\item Interpretacja metadanych ICY i~wyświetlanie tytułów utworów na ekranie,
				\item Lista ulubionych
			\end{itemize}
			
			Oprócz powyższych ulepszeń, odnaleziono i~naprawiono szereg błędów takich jak nieprawidłowe zakańczanie odtwarzania i~w konsekwencji zawieszanie się programu, niemożność wyjścia z~ekranu ładowania wyników wyszukiwania czy dereferencja zerowego wskaźnika w~zadaniu wypisywania statystyk po zakończeniu odtwarzania.
			
		\subsection{Ostateczna wersja prototypu}
			Dysponując prawie w~pełni funkcjonalnym oprogramowaniem układowym działającym na rozwojowej wersji prototypu przystąpiono do prac nad ostateczną wersją prototypu części sprzętowej odbiornika.
			$ $\\
			
			Opracowano schemat ideowy bazujący na konstrukcji wersji rozwojowej. Wprowadzono również pewne  zmiany i~ulepszenia. Oto niektóre z~nich:
			\begin{itemize}
				\item Prąd chwilowy pobierany przez wzmacniacz przekroczył możliwości przetwornicy, układ zasilony został z~niższego napięcia bezpośrednio z~ogniw,
				\item Dodany został pomiar napięcia baterii,
				\item Do wzmacniacza doprowadzony został sygnał wyciszający, co pozwoliło zmniejszyć uciążliwość szumów pochodzących od niedostatecznie filtrowanego napięcia zasilającego, szczególnie słyszalne przy braku innych dźwięków.
			\end{itemize}
			
			Na podstawie schematu zaprojektowano mozaikę ścieżek obwodu drukowanego, który następnie wykonano wykorzystując metodę termotransferową \cite{ch4_pcb_method}.
			Po przeniesieniu wydruku ścieżek na laminat pokryty warstwą miedzi, zostały one wytrawione roztworem nadsiarczanu sodu. Następnie zostały wywiercone otwory pod montaż przewlekany. Na rysunku~\ref{4/pcb} przedstawiono laminat po wywierceniu kilku pierwszych otworów.
			$ $\\
			
			Po przetestowaniu kompletnego prototypu zmontowanego na płytce PCB (rysunek~\ref{4/prototype_3}), zaprojektowana i~wykonana została pierwsza wersja obudowy. Panel frontowy pokazany został na rysunku~\ref{4/case1}.

			\imghss{4/pcb}{Płytka PCB po wytrawieniu, początek procesu wiercenia otworów}{4/prototype_3}{Kompletny prototyp na płytce}
			
			Po zmontowaniu całości okazało się, że gwinty w~niektórych otworach słabo trzymają wkręcone w~nie śruby co powoduje utratę sztywności obudowy i~trwałości mocowania komponentów. Do projektu obudowy wprowadzono niezbędne poprawki oraz dodano uchwyt ułatwiający przenoszenie odbiornika. Zmontowane urządzenie zamknięte w~drugiej wersji obudowy pokazane jest na rysunku~\ref{4/case2}.
			
%			Powstały dwie wersje obudowy. Druga poprawiła największy mankament wersji pierwszej, mianowicie bardzo słabo trzymające gwinty. Na podstawie wydruków testowych stwierdzono, że otwór o~średnicy 2.8mm jest najlepiej dopasowany do wkrętów M3x12. Dodatkowo zaprojektowano uchwyt do podnoszenia radia. Kompletna obudowa w~wersji drugiej widoczna jest na 
			
			\imghss{4/case1}{Panel frontowy obudowy}{4/case2}{Kompletna obudowa}
			
			W~sierpniu radio było już gotowe. Po nieznacznych korektach w~oprogramowaniu (np. zmiana numerów wyprowadzeń) było produktem gotowym do testowania przez osoby postronne.

		\subsection{Testowanie}
			Odbiornik przekazano do użytkowania osobom, które można określić mianem standardowego użytkownika.
			Pierwszym z~zauważonych problemów był zbyt krótki czas oczekiwania na połączenie z~serwerem stacji i~odbiór danych. Radio dotychczas było testowane w~bezpośrednim sąsiedztwie routera Wi-Fi. Po przeniesieniu do innego pomieszczenia radio nie mogło połączyć się z~większością stacji z~powodu zbyt szybkiego zgłaszania błędu przekroczenia czasu oczekiwania na dane.
			Powiększono zatem czas oczekiwania do 4~sekund, co rozwiązało problem.
			$ $\\
			
			Użytkowanie radia jako kompletnego produktu szybko dostarczyło nowych danych o~błędach. Dodana została obsługa MPEG2 (obsługa niższej jakości dźwięku), ponieważ niektóre stacje nie odtwarzały się poprawnie. Ograniczona została liczba przekierowań, na które odpowiada klient HTTP, ponieważ przy ich zbyt dużej liczbie następowało zawieszanie się urządzenia.
			Wykryto i~naprawiono również problem z~odbiorem metadanych w~formacie ICY. Pierwotnie ich długość była zapisywana do zmiennej jednobajtowej co w~przypadku stacji nadającej więcej informacji niż tylko tytuł utworu powodowało zawieszanie się odtwarzania. Po zwiększeniu rozmiaru zmiennej do 4~bajtów problem został rozwiązany.
			$ $\\
			
			To tylko niektóre z~błędów napotkanych przez osoby z~najbliższego otoczenia. Wszystkie odnalezione były niezwłocznie zapisywane w~systemie do śledzenia zadań i~później naprawiane.
%			Sierpień upłynął, radio zyskiwało na użyteczności.
			
		\subsection{Odtwarzanie plików lokalnych}
			Jako że funkcjonalność odtwarzania plików z~karty SD była wymaganiem opcjonalnym, zrealizowana została pod koniec prac nad urządzeniem. We wrześniu został dodany ekran listowania plików. Za ekran wykonawczy posłużył ekran radia z~pomniejszymi zmianami. Dodano pasek postępu informujący użytkownika jaka część utworu została już odtworzona. Znalazło się też miejsce na informacje o~czasie trwania całości. Oba ekrany przedstawiono na rysunkach \ref{4/scr_radio} oraz \ref{4/scr_player}.
			
			\imghss{4/scr_radio}{Ekran radia}{4/scr_player}{Ekran odtwarzacza lokalnego}
			
			Aby móc wyświetlać informacje o~autorze i~tytule odtwarzanego utworu, zaimplementowana została obsługa formatu metadanych ID3v2.
			Radio w~pierwszej połowie września zyskało interfejs do odtwarzania plików lokalnych.
%			Dalej, aż do przełomu listopada i~grudnia tworzona była niniejsza dokumentacja.
			
		\subsection{Automatyczne połączenie Wi-Fi}
			Funkcjonalność ta nie była wymagana w~żaden sposób do testów, stąd jej priorytet podczas prac nad projektem był relatywnie niski. 
			Do czasu wprowadzenia tej funkcjonalności parametry dostępu do sieci Wi-Fi były zapisane na stałe w~kodzie programu.
			Jednak aby radio mogło pracować w~rożnych miejscach (korzystać z~różnych sieci Wi-Fi), konieczne było
			zaimplementowanie funkcjonalności dodawania i~konfigurowania przez użytkownika parametrów dostępowych do różnych sieci, ich przechowywania w~urządzeniu oraz automatyzacja procesu łączenia się z~najlepszą siecią podczas startu urządzenia.
			W~pierwszej połowie grudnia w.w. funkcjonalność była gotowa.
			 
	 \section{Organizacja pracy}
		Praca była tworzona przez jedną osobę.
		Do planowania i~organizacji pracy nad projektem użyto oprogramowania Jira. Główną wykorzystaną funkcjonalnością było tworzenie zadań.
		Następnie były one grupowane w~bloki, których wykonanie skutkowało powstaniem wybranej, elementarnej funkcjonalności, takiej jak na przykład odtwarzanie lokalnych plików czy automatyczne połączenia Wi-Fi.
		Bloki te zazwyczaj były następnie uruchamiane jako 1-2 tygodniowe sprinty. Przykłady takich bloków zostały przedstawione na rysunku~\ref{4/plan2}. Bloki ulepszeń (rys.~\ref{4/plan1}) czy błędów, istniały ciągle, dopisywane były do nich nowe pomysły czy zauważone błędy. Następnie mogły być one przenoszone do innych bloków.
		W~ramach jednego sprintu wykorzystywana była tablica Kanban.
 		\imghss{4/plan1}{Blok ulepszeń}{4/plan2}{Zgrubny zarys dokumentacji, lokalnego odtwarzania i~części sprzętowej}
		 
		 Utworzono szereg własnych typów zadań. Zostały one przedstawione w~tabeli~\ref{jicons}.
		 
		 \begin{table}[H]
		 	\centering
		 	\caption{Typy zadań w~oprogramowaniu Jira}
		 	\label{jicons}
			 \newcommand{\jicon}[1]{\includesvg[width=12px]{img/4/jira/#1}}
			 \begin{tabular}{l|l|l}
			 	\jicon{bug}&\textit{Bug}&niedziałająca funkcjonalność\\
			 	\jicon{ref}&\textit{Refactor}&głębokie zmiany w~strukturze projektu\\
			 	&&\\
			 	
			 	\jicon{docs_story}&\textit{[Docs] Story}&blok dokumentacji\\
			 	\jicon{research}&\textit{Research}&konieczność zbadania sytuacji\\
			 	\jicon{doc}&\textit{Documentation}&udokumentowanie działania\\
			 	&&\\
			 	
			 	\jicon{story}&\textit{Story}&blok zadań\\
			 	\jicon{impr}&\textit{Improvement}&ulepszenie istniejącej funkcjonalności\\
			 	\jicon{task}&\textit{Task}&nowe funkcje\\
			 	&&\\
			 	
			 	\jicon{hw}&\textit{Hardware}&poprawka sprzętowa\\
			 \end{tabular}
	 	\end{table}
		 
		Sposób organizacji pracy przy budowie urządzenia i~wytworzeniu jego oprogramowania układowego był  inspirowany był metodyką Scrum. Została ona jednak znacznie uproszczona w~związku z~realizacją projektu przez jedną osobę.
	 	
\cleardoublepage
\chapter{Wyniki projektu}
	W~tym rozdziale omówione zostały wyniki pracy.
	Projekt podsumowano i~wyciągnięto wnioski.
%	Projekt został podsumowany i~wyciągnięte zostały wnioski
%	skupiono się na finalnym efekcie projektu.
	
	\section{Przegląd zrealizowanych funkcji}
		Główną funkcją produktu jest odtwarzanie radia internetowego. Odtwarzacz został zrealizowany w~postaci dedykowanego ekranu przedstawionego na rysunku~\ref{5/radio}. Urządzenie prezentuje na nim podstawowe informacje na temat stacji radiowej oraz stanu odtwarzacza. Obsługa metadanych ICY pozwala na wyświetlanie tytułu i~wykonawcy granego w~danej chwili utworu. Poniżej wyświetlane są statystyki działanie dekodera. \textit{CPU} odnosi się do procentu wykorzystania rdzenia dekodującego, natomiast \textit{Bufor} informuje użytkownika o~poziomie wypełnienia bufora (przydatne, gdy występują problemy z~odtwarzaniem spowodowane niską jakością połączenia internetowego).
		
		\imghss{5/radio}{Ekran \textit{Radio} (włączone z~listy ulubionych)}{5/radio_search}{Ekran \textit{Radio} (wyszukane)}
	
		\paragraph{Dalszy rozwój}
			Implementacja dekodowania formatu AAC/AAC+ znacząco podniosłaby użyteczność produktu.
			
		\subsection{Lista ulubionych stacji}
			Przechowywana w~pamięci urządzenia lista ulubionych stacji zawiera znalezione i~zapisane przez użytkownika stacje. Przedstawia ją rysunek~\ref{5/fav}.
			Ekran obsługuje stronicowanie wpisów. Domyślnie wyświetlanych jest 16~wpisów na stronę. Na dole ekranu wyświetlona jest aktualna strona i~liczba wszystkich stron.
			W~górnym pasku zawiera się nazwa ekranu oraz ikony statusowe (Wi-Fi, karta SD, bateria).
			
			\imgh{5/fav}{Ekran \textit{Ulubione stacje}}{0.5}
			
			Jest to także ekran startowy. Wyświetla się on jako pierwszy po starcie programu. Dlatego jest też traktowany jako ekran główny i~posiada najwięcej ikon akcji na dole ekranu.
			Ikona najbardziej po lewej to ikona folderu służąca do uruchamiania odtwarzacza plików lokalnych. Zaraz obok znajduje się ikona ustawień. Po prawej użytkownik może przejść do wyszukiwarki stacji.
			
			\paragraph{Dalszy rozwój}
				Warto zwrócić uwagę, że nowe stacje są zawsze dodawane na koniec listy ulubionych. Nie ma żadnej możliwości zmiany ich ustawienia. Opcja ręcznego sortowania listy ulubionych znacząco podniosłaby komfort użytkowania produktu. Przydatną opcją byłaby także lista ostatnio odtwarzanych stacji. Możliwość importu/eksportu listy ulubionych stacji z/na kartę SD umożliwiłaby tworzenie kopii zapasowych listy ulubionych stacji lub też ręczne dodawanie strumieni których nie ma w~obsługiwanych wyszukiwarkach stacji internetowych.
			
		\subsection{Wyszukiwarka stacji radiowych}
			Wyszukiwanie stacji jest dwuetapowe. Najpierw użytkownik na ekranie z~klawiaturą (rys.~\ref{5/search}) podaje fragment nazwy stacji, którą chciałby wyszukać. Po kliknięciu znajdującej się w~prawym dolnym rogu ekranu ikony lupy, ładowane są stacje pasujące do wyszukiwanego wzorca. Pojawia się nowy, przedstawiony na rysunku~\ref{5/searchres} ekran, na którym wyświetlają się wyszukane stacje.
			
			\imghss{5/search}{Ekran wyszukiwania stacji}{5/searchres}{Ekran wyników wyszukiwania}
			
			Analogicznie do ekranu ulubionych, tutaj też zawarto obsługę stronicowania wyników. Jednak nie jest dostępna liczba wszystkich wyników, stąd ikona nieskończoności na dole ekranu.
			
			\paragraph{Dalszy rozwój}
				Obecnie obsługiwany jest tylko jeden dostawca wyszukiwania. Istnieje infrastruktura do obsługi wielu takich serwerów, lecz żaden inny nie został dodany.
			
		\subsection{Przeglądarka i~odtwarzacz plików lokalnych}
			Listing plików lokalnych odczytanych z karty SD przedstawiony został na rysunku~\ref{5/local}.
%			Tworzony jest on na podstawie plików znajdujących się na karcie SD.
			Zapewnione jest sortowanie wpisów (foldery sortowane są jako pierwsze). Możliwe jest rekurencyjne przeglądanie drzewa katalogów. Nazwa ekranu jest dynamicznie aktualizowana i~wyświetla nazwę aktualnego folderu. Jak widać obsługiwane jest także przewijanie długich tekstów poziomo na ekranie oraz dekodowanie metadanych ID3v2. Utwory w~ramach jednego folderu są automatycznie odtwarzane jeden po drugim. Interfejs lokalnego odtwarzacza można zobaczyć na rysunku~\ref{5/local_play}.
			
			\imghss{5/local}{Ekran \textit{Pliki lokalne}}{5/local_play}{Odtwarzacz plików lokalnych}
			
			\paragraph{Dalszy rozwój}
				Przydatnym dodatkiem byłaby możliwość randomizacji kolejności automatycznego odtwarzania utworów oraz opcja odtworzenia całego folderu rekurencyjnie. Pozwoliłoby to np. na losowe odtworzenie całej dyskografii jednego zespołu.
				Implementacja bezstratnego kodowania plików FLAC zapewniłaby większą kompatybilność urządzenia oraz potencjalne podniesienie jakości dźwięku.
		
		\subsection{Połączenie z~Internetem}
			Urządzenie do odtwarzania radia i~wyszukiwania nowych stacji wymaga połączenia z~Internetem. W~aktualnej wersji oprogramowania zostały zaimplementowane funkcje wyszukiwania, zapisywania oraz łączenia się z~sieciami Wi-Fi.
			$ $\\
			
			Dane dostępowe do znalezionych sieci z którymi udało się połączyć są zapisywane w~pamięci nieulotnej odbiornika. Urządzenie po uruchomieniu automatycznie wyszukuje najlepszą sieć, do której hasło zostało wcześniej zapisane. Dzięki temu, użytkownik od razu może rozpocząć odtwarzanie radia.
	
	\section{Scenariusze użytkowania}
		W~tej sekcji przedstawione zostały scenariusze użytkowania urządzenia.
	
		\subsection{Wyszukanie i~zapisanie stacji}
			\newcommand{\rfladd}[1]{\ref{5/PicoRadio-fl-add}#1}
			
			Na rysunku~\rfladd{} przedstawiono podstawowy scenariusz wyszukania stacji i~dodania jej do listy ulubionych.
			
			\imgh{5/PicoRadio-fl-add}{Użytkownik dodaje stację do ulubionych}{0.9}
			
			\begin{itemize}
				\item Rys.~\rfladd{a} -- użytkownik uruchomił radio i~zmienił pozycję kursora na przycisk akcji odpowiedzialny za wyszukiwanie,
				\item Rys.~\rfladd{b} -- po naciśnięciu środkowego przycisku, wyświetlony został ekran wyszukiwania z~klawiaturą i~wpisana została fraza filtrująca,
				\item Rys.~\rfladd{c} -- na 3. stronie odnaleziona została żądana stacja
				\item Rys.~\rfladd{d} -- po akceptacji wyboru (środkowy przycisk), rozpoczęło się odtwarzanie; użytkownik przeniósł kursor na przycisk gwiazdki,
			\end{itemize}
			Warto nadmienić, że w~każdej chwili możliwa jest korekta wyboru. Gdy jednak znaleziona stacja nie jest odpowiednia lub popełniono błąd we frazie wyszukiwania, użytkownik może cofnąć się przy użyciu skrajnie lewej ikony akcji do poprzedniego ekranu.
			\begin{itemize}
				\item Rys.~\rfladd{e} -- po kliknięciu przycisku dodania do ulubionych zmienia on wypełnienie (stacja została zapisana),
				\item Rys.~\rfladd{f} -- użytkownik chce szybko powrócić do listy ulubionych stacji, więc przy pomocy prawego przycisku zmienia pozycję kursora na przycisk szybkiego powrotu,
				\item Rys.~\rfladd{g} -- nowo dodana stacja automatycznie jest wyświetlona i~zaznaczona; użytkownik dostaje potwierdzenie, że zapis powiódł się.
			\end{itemize}
			
		\subsection{Odtwarzanie plików lokalnych}
			\newcommand{\rflocal}[1]{\ref{5/PicoRadio-fl-local}#1}
			
			Na rysunku~\rflocal{} przedstawiono scenariusz otwierania kolejnych lokalnych folderów i~odtworzenia zapisanego na karcie SD pliku MP3.
			
			\imgh{5/PicoRadio-fl-local}{Użytkownik odtwarza lokalny plik MP3}{0.9}	
			
			\begin{itemize}
				\item Rys.~\rflocal{a} -- użytkownik ustawił kursor na ikonie akcji odtwarzacza lokalnego,
				\item Rys.~\rflocal{b} -- po przewinięciu w~dół (pasek przewijania widoczny po prawej stronie ekranu) wskazano żądany podfolder,
				\item Rys.~\rflocal{c} i~\rflocal{d} -- zatwierdzając wybór użytkownik przechodzi do podfolderu; jego nazwa wyświetla się na górze ekranu,
				\item Rys.~\rflocal{e} -- klikając w~plik, odtwarzanie rozpoczyna się.
			\end{itemize}
			
			Co ciekawe, pliki lokalne traktowane analogicznie do stacji radiowych, to znaczy można je dodawać do listy ulubionych.
			
		\subsection{Połączenie z~nową siecią Wi-Fi}
			\newcommand{\rflwifi}[1]{\ref{5/PicoRadio-fl-wifi}#1}
			
			Na rysunku~\rflwifi{} przedstawiono scenariusz łączenia się z~nową siecią Wi-Fi i~zapisanie jej do znanych sieci.
			
			\imgh{5/PicoRadio-fl-wifi}{Użytkownik łączy się z~nową siecią}{0.9}
			
			\begin{itemize}
				\item Rys.~\rflwifi{a}, \rflwifi{b}, \rflwifi{c} i~\rflwifi{d} -- użytkownik przechodzi do wyszukiwania nowych sieci i~znajduję tą, z~którą chce się połączyć,
				\item Rys.~\rflwifi{e}, \rflwifi{f} -- otwiera się ekran wprowadzania hasła; rysunki prezentują działanie pierwszego z~dwóch trybów działania przycisku \textit{Shift} (tymczasowe działanie),
				\item Rys.~\rflwifi{g}, \rflwifi{h} -- tutaj, poprzez podwójne naciśnięcie, został wybrany drugi tryb \textit{Shift} (stałe działanie),
				\item Rys.~\rflwifi{i}, \rflwifi{j} -- po wprowadzeniu hasła, otwiera się ekran połączenia, na którym prezentowany jest aktualny status połączenia i~ewentualne błędy,
				\item Rys.~\rflwifi{k} -- gdy połączenie się powiedzie, przycisk wstecz przenosi użytkownika do listy zapisanych sieci (zamiast do poprzedniego ekranu),
				\item Rys.~\rflwifi{l} -- użytkownik może teraz z~powodzeniem słuchać radia korzystając z~nowej sieci Wi-Fi.
			\end{itemize}
			
	\section{Zrealizowane oprogramowanie układowe}
			Oprogramowanie układowe sprawia, że odbiornik spełnia wszystkie wymagania sformułowane w założeniach.
			Udało się zaimplementować dekodowanie MP3 z~każdą przepływnością, nawet 320~kbps, w~wariantach MPEG1 oraz MPEG2, a także odtwarzanie plików w formacie WAV.
			Zrealizowana została obsługa metadanych zarówno strumieni internetowych (ICY) jak i~lokalnych plików MP3 (ID3v2). Dekoder potrafi uwzględniać i~korygować wiele nieprawidłowości w~strumieniu danych. Obsługiwane jest także wykrywanie i~zgłaszanie błędów dekodowania. Klient HTTP obsługuje minimum funkcji, lecz jest to wystarczające do poprawnego odbioru strumienia radia internetowego.
			$ $\\
			
%			Należałoby poprawić dokładność wyliczania procentu pozostałej energii w~ogniwach. Aproksymacja zgrubnie jest poprawna lecz do pełnej poprawności brakuje wielogodzinnych testów i~poprawek. Podczas ładowania, wskazanie poziomu baterii jest znacznie zawyżone.
%			$ $\\
%			
%			Błędy zgłaszane użytkownikowi są zbyt ogólne. Konieczna jest unifikacja kodów błędów w~projekcie i~wyświetlanie adekwatnej wiadomości zamiast ogólnikowego \textit{Wystąpił błąd}. Brak jest także ekranu potwierdzenia usunięcia wpisu z~ulubionych oraz edycji nazwy zapisanej stacji.

			Pośród zrealizowanych funkcjonalności są takie, które działają, ale wymagałyby dalszego dopracowania. Jedną z nich jest określanie stopnia naładowania wbudowanego ogniwa. Obecnie szacowanie pojemności w trakcie rozładowywania jest poprawne ale mało dokładne, natomiast podczas ładowania wskazania są mocno zawyżone. 
			Drugim zagadnieniem jest uporządkowanie podsystemu zgłaszania przez oprogramowanie błędów w działaniu urządzenia. Konieczna jest unifikacja kodów błędów i rozbudowa systemu powiadamiania o nich użytkownika.
			Pewnego dopracowania wymaga również funkcjonalność operowania listą ulubionych. Aktualnie brakuje ekranu potwierdzenia usunięcia pozycji z listy oraz możliwości edytowania nazwy stacji znajdującej się na tej lisie.
		
		\subsection{Perspektywa nowej generacji projektu}
			Swoista wersja 2.0 oprogramowania układowego mogłaby zawierać dodatkowe funkcjonalności, które można by zrealizować bez wprowadzania zmian w części sprzętowej odbiornika.
			$ $\\
			
			Istniejąca architektura oprogramowania pozwala na zaimplementowanie funkcjonalności tworzenia więcej niż jednej listy ulubionych. Nie zostało to jednak obecnej wersji zrealizowane. Dodanie takiej możliwości z pewnością uatrakcyjniło by korzystanie z odbiornika pozwalając użytkownikowi grupować utwory czy stacje według gatunków muzycznych, wykonawców czy też innych kryteriów.
			Bardzo atrakcyjne wydaje się dodanie możliwości obsługi modemów USB pozwalających na korzystanie z sieci komórkowych LTE/5G. Modem taki zapewniałby dostęp do Internetu praktycznie w dowolnym miejscu a to pozwoliłoby to korzystać z urządzenia w sposób niemal identyczny jak korzysta się z tradycyjnego odbiornika radiowego, bez potrzeby podłączania się do lokalnej sieci Wi-Fi. Trzecia, warta rozważenia funkcjonalność wiąże się z faktem, że cześć stacji internetowych w momencie nawiązywania połączenia próbuje wymusić przejście na połączenie szyfrowane z użyciem protokołu TLS. Obecnie odbiornik odrzuca takie prośby i zgłasza błąd. W nowej wersji oprogramowania należało by rozważyć zaimplementowanie obsługi takich połączeń. 
			
	\section{Część sprzętowa}
		Projekt od strony sprzętowej został udokumentowany w~sekcji~\ref{sec:hw}. Oprócz schematu, powstała także płytka PCB, oraz obudowa wykonana w technice druku~3D. Całość widoczna jest na rysunku~\ref{5/hw_result}.
		$ $\\
		
		Od strony funkcjonalnej urządzenie posiada wbudowane ogniwa, dzięki czemu nie potrzebuje zewnętrznego zasilania. Zintegrowany układ ładowania ze złączem USB~C pozwala na wykorzystanie praktycznie dowolnej ładowarki z~tym złączem (prąd wykorzystany do ładowania to 1A). Produkt posiada analogową regulację głośności potencjometrem wyprowadzonym na lewą stronę urządzenia. Lokalizacja została dobrana ze względu na bliskość sekcji audio na płytce i~chęć minimalizacji długości ścieżek, które im dłuższe tym większe zakłócenia wprowadzają do systemu. Obok gałki potencjometru znajduje się także złącze JACK~3.5mm przeznaczone do podłączenia zewnętrznego wzmacniacza.
		
		\imgh{5/hw_result}{Gotowy produkt w~obudowie}{0.7}
		
		\subsection{Ocena wyników}
%			Prawie wszystkie wykryte we wcześniejszych wersjach zostały naprawione w trzeciej wersji prototypu odbiornika.
			Sprzętowa natura projektu spowodowała, że znacznie trudniej było korygować powstałe błędy.
			Ta sekcja zawiera listę niedociągnięć, których wyeliminowanie umożliwiłoby w przyszłości stworzenie lepszego produktu.

				\begin{itemize}
					\setlength{\itemsep}{0.5cm}
					\item Błędy w schemacie elektrycznym:
					\begin{itemize}
						\setlength{\itemsep}{0.2cm}
						\item brak kondensatorów filtrujących, które mogłyby poprawić parametry pracy niektórych modułów,
						\item niedokładne rezystory pomiarowe w~dzielnikach napięć powodują odchylenia w~pomiarze napięć \lstinline|VCC| i~ogniw,
					\end{itemize}

					\item Błędy w projekcie obwodu drukowanego:
					\begin{itemize}
						\setlength{\itemsep}{0.2cm}
						\item brak otworów montażowych w~obrębie głównej części płytki, przez co posiada ona dość niestandardowy kształt,
						\item płytka zamontowana jest w rogu obudowy, uniemożliwiając umieszczenie w tym miejscu elementów łączących fragmenty obudowy, co powoduje jej niestabilność,
						\item podłączanie modułów do głównej płytki z wykorzystaniem złączy typu goldpin, jest zawodne, gdyż może to skutkować ich wypadnięciem lub utratą kontaktu elektrycznego.
					\end{itemize}
				\end{itemize}
				
				Dodatkowo, podczas testowania wykryte zostały problemy, których nie udało się poprawić w~tej wersji produktu.
				Parametr $\text{RDS}_{\text{(on)}}$ tranzystora \lstinline|Q1| przy małych napięciach jest relatywnie duży, co przy znacznym poborze prądu przez wzmacniacz powoduje odkładanie się na nim nietrywialnej mocy. Przykładowym rozwiązaniem tego problemu może być zastosowanie tranzystora o~lepszych parametrach, lub zrezygnowanie wbudowanej ładowarki ogniw.
				Niewystarczające filtrowanie zasilania powoduje słyszalne szumy, szczególnie wtedy, kiedy nic nie jest odtwarzane. Komunikacja Wi-Fi powoduje znaczny chwilowy wzrost prądu pobieranego przez płytkę Pi~Pico~W, co powoduje zakłócenia w~zasilaniu wzmacniacza. Częściowo zostało to naprawione przez wyciszanie wzmacniacza, lecz nie jest to rozwiązanie idealne. Należałoby odseparować od siebie te dwa obwody zasilające.
				$ $\\
				
%			\subsubsection{Płytka PCB}
%				Przy projektowaniu płytki PCB również wystąpiły niedopatrzenia:
%				\begin{itemize}
%					\item brak otworów montażowych w~obrębie głównej części płytki, przez co posiada ona dość niestandardowy kształt,
%					\item brak miejsca na elementy spajające obudowę, tam gdzie płytka dotyka ścianek nie ma wkrętów mocujących, co powoduje niestabilność obudowy,
%					\item moduły są wpinane w~złącza typu \textit{goldpin}, co w~przypadku obecności na nich złącz zewnętrznych (JACK, port USB~C), powoduje że wpinanie wtyków w~te złącza odchyla je i~w konsekwencji moduł wypada lub traci kontakt. Zostało to częściowo naprawione przy module ładowania. Wlutowany na stałe, stał się stabilniejszy.
%				\end{itemize}
				
				Metoda termotransferu nie zapewnia najlepszych parametrów produkcyjnych. Niektóre ścieżki są cieńsze niż w projekcie, a~część padów lutowniczych jest tak cienka, że montaż był znacząco utrudniony. Lepszym i~powtarzalnym sposobem na wytwarzanie płytek drukowanych byłoby zlecenie ich wykonania zewnętrznej firmie.
		
		\subsection{Radio nowej generacji}
			Pierwszym ulepszeniem powinno być skorygowanie schematu oraz stworzenie nowej płytki, z zaprojektowaną od podstaw mozaiką ścieżek. Wykonana profesjonalnie, dwuwarstwowo, pozwoliłaby na poprawienie parametrów pracy, głównie obniżenie szumów sekcji zasilającej. Powinny zostać uwzględnione otwory montażowe płytki.
			Złącza zamontowane na modułach nie powinny być wykorzystywane.
			Podłączanie do nich zewnętrznych przewodów, wywiera nadmierne naprężenia na ich zawodne połączenie z resztą obwodu.
			W miejsce tych złączy powinny zostać wyprowadzone ich lutowane lub przykręcane odpowiedniki.
			$ $\\

			Należałoby się zastanowić także, czy jakość i poziom głośności dźwięku odpowiada użytkownikowi. Jeżeli nie, istnieje możliwość wykorzystania mocniejszego wzmacniacza w~celu obniżenia zniekształceń przy cichym słuchaniu lub zwiększenia mocy maksymalnej.
			Należy jednak mieć na uwadze, że układy o~większej mocy wymagają wyższego napięcia zasilającego.
			Przykładowo PAM8610\textsuperscript{\cite{ch5_pam8610}} przy napięciu 7V (przybliżone napięcie 2~ogniw połączonych szeregowo) jest w~stanie oddać moc 4W na kanał (obecnie użyty wzmacniacz posiada moc ok. 1W na kanał). Nowszy PAM8006a\textsuperscript{\cite{ch5_pam8006a}} jest w~stanie wysterować głośniki mocą aż 8W na kanał, lecz wiąże się to z~napięciem zasilania 12V (3 ogniwa szeregowo) i~zmianą głośników na takie o~impedancji $8 \Omega$.
			Podwyższenie napięcia zasilającego niesie ze sobą konieczność wprowadzenia poważnych zmian w budowie sekcji zasilającej. Połączone szeregowo ogniwa litowo-jonowe wymagają znacznie bardziej zaawansowanych układów zabezpieczających i ładujących. Bardziej skomplikowane byłoby również wykorzystanie złącza USB-C do ładowania i zasilania urządzenia (należałoby zastosować moduł ładowarki obsługujący funkcjonalność USB-PD\textsuperscript{\cite{usb_pd}}).
			$ $\\
			
			Dodatkowym modułem poprawiającym jakość dźwięku mógłby być prosty \textit{equalizer}. Dwa pokrętła do wyrównania pasma akustycznego (bas i~treble), zrealizowane z~wykorzystaniem pasywnych elementów elektronicznych.
			
	\section{Podsumowanie}
	
		W~niniejszej pracy inżynierskiej udało się zaprojektować i~zbudować odbiornik internetowych stacji radiowych z~opcją odtwarzania plików lokalnych.
		Wszystkie założone funkcjonalności zostały zrealizowane. Pomimo kilku niedociągnięć urządzenie jest w~pełni funkcjonalne i~z powodzeniem może być wykorzystane do odtwarzania radia internetowego i~lokalnych plików MP3 oraz WAV. Powstała też założona obudowa, dzięki której radio nabiera wyglądu kompletnego produktu.
		
%		Projekt uważam za udany. 
%		Z~efektów pracy jestem nad wyraz zadowolony. Urządzenie da się wykorzystać do codziennego słuchania radia czy muzyki.
%%		Wbudowane ogniwa zapewniają przenośność urządzenia
%		Klient (promotor) również jest zadowolony. Temat był propozycją promotora. Zostały zrealizowane, a~nawet przekroczone wszystkie jego założenia. Ponad wymagania zrealizowano projekt płytki czy obudowy, w~której został zamknięty finalny produkt.
%		$ $\\
%		
%		Dużo się nauczyłem.
%		Znacząco rozwinąłem swoje umiejętności programowania w~języku C++, szczególnie w~zakresie tworzenia i~utrzymywania średniego rozmiaru projektów. Zaprezentowane hierarchę klas były implementowane i~refaktoryzowane przez kilka miesięcy. Pozwoliło mi to na doskonalenie dobrych praktyk programistycznych.
%		
%		$ $\\
%		Druk 3D był dla mnie nowym obszarem zainteresowań. Udało mi się dobrać odpowiedni program, zaprojektować i~wydrukować obudowę projektu. 
%		Mój warsztat został doposażony, drukarka na pewno okaże się pomocna w~przyszłych projektach.
			
	% TODO warnings gdzieś przy flow??
	% TODO instrukcja obsługi (rozdział 5) ??
	
\cleardoublepage
\phantomsection
\addcontentsline{toc}{chapter}{Bibliografia}
	\printbibliography

\cleardoublepage
\phantomsection
\addcontentsline{toc}{chapter}{Spis rysunków}
	\listoffigures

\cleardoublepage
\phantomsection
\addcontentsline{toc}{chapter}{Załączniki}
\chapter*{Załączniki}
%	abc
	\begin{enumerate}[label=Załącznik \arabic*, itemsep=0.5cm, leftmargin=3.0cm]
		\item (folder \lstinline|kicad|) -- Projekt elektryczny odbiornika w~programie KiCAD. Zawiera pliki schematu oraz płytki PCB.
		\item (folder \lstinline|3d|) -- Projekt obudowy 3D w~programie OpenSCAD oraz wygenerowane pliki \lstinline|.stl| poszczególnych elementów urządzenia gotowe do wykorzystania w~programie typu \textit{slicer}.
		\item (folder \lstinline|pico-radio-cpp|) -- Kod źródłowy oprogramowania układowego oraz plik \lstinline|.uf2| gotowy do wgrania do urządzenia.
	\end{enumerate}

	
\end{document}
